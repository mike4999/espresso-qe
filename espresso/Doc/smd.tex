\documentclass[aps,prb,preprint,groupedaddress]{revtex4}
\usepackage{graphicx}
\usepackage{latexsym}


%------------------------------------

\begin{document}

% Title
\maketitle

%--------------------------------------------------

\normalsize

% *************************************************************************************
\noindent
{\large \bf How to use SMD option in CP code to find MEP and TS. (as of 08/02/05)}
\vskip 50 pt
\noindent
Yosuke Kanai (ykanai@princeton.edu) 
\vskip 50 pt

\noindent
In this file, I will explain how one can use SMD option of CP code to locate
the minimum energy path (MEP) and the transition state (TS), given one knows
the initial and final states of a transition.

\vskip 30 pt
\noindent
SMD option is used to invoke String Method, which can be used to either locate
the MEP or TS of a transition. The method was first proposed by Prof. W.E and 
his collaborators in Phys. Rev. B 66, 052301 (2002), 
and it was extended for implementation with the
DFT in context of Car-Parrinello formulations in J. Chem. Phys. 121, 3359 (2004).
The current implementation closely follows that of the JCP paper.
Please read the JCP paper in order to understand how it works, and the more
mathematical backgrounds can be found in the PRB paper and Ph.D thesis of Weiqing Ren (Math, Princeton).
The method has a very close resemblance to the Nudged Elastic Band (NEB) method.

\vskip 50 pt
\noindent
{\Large \bf Limitations }  
\vskip 5 pt
\noindent
1. Only $\Gamma$ point sampling in BZ integration is possible as in the CP code. \\
2. It is not compatible with the variable cell dynamics (Any ideas ?). \\ 
3. Finite-temperature version is not implemented. 

\newpage
\noindent
{\Large \bf Overview : How it works }
\vskip 10 pt
\noindent
{\bf Step 0: Familiarize yourself with CP code.} \\
\vskip 5 pt
\begin{minipage}{6in} 
\leftskip1.5cm
Being familiar to regular Car-Parrinello approaches (CP code) is
{\bf essential} before using the String method to study transitions. 
It is {\bf important} especially to be familiar with the damped dynamics optimization.
\end{minipage}

\vskip 30 pt
\noindent
{\bf Step 1: Locate the initial and final configrations (minima).} \\
\vskip 5 pt
\begin{minipage}{6in}
\leftskip1.5cm
This can be done straight-forwardly using CP code etc.
It is however important to make sure that you have ``good" minima, especially
when studying exotic systems where our chemical intuitions (or physical insights)
are not very reliable. I find doing some MD runs help in many cases to identify
``good" minima.
\end{minipage}

\vskip 30 pt
\noindent
{\bf Step 2: Optimize the electronic degrees of freedom for the replicas of string.} \\
\vskip 5 pt
\begin{minipage}{6in}
\leftskip1.5cm
The initial trial path is generally a linear interpolated path between the 
initial and final points of the transition. If one knows some other points of the
transition, those can be incorporated to obtain a better initial guess for the path.
Make sure that no atoms in any of replicas overlap with each other (especially if 
the transition is an exchange reaction).  
\vskip 20 pt
num\_of\_images \\
smd\_smcp  \\
smd\_linr $or$ smd\_polm \\
smd\_kwnp \\
smd\_ene\_ini $and$ smd\_ene\_fin \\
\end{minipage}

\vskip 30 pt
\noindent
{\bf Step 3: Optimize the string (replicas) to locate the MEP. } \\
\vskip 5 pt
\begin{minipage}{6in}
\leftskip1.5cm
Before you start, make sure the electronic degrees of freedom are sufficiently
optimized so that ionic forces are meaningful. 
The replicas are now optimized simultaneously to locate the MEP using the
damped dynamics.
\vskip 20 pt
smd\_smcp = .FALSE. \\
smd\_smlm \\
smd\_lmfreq \\
smd\_tol \\
smd\_maxlm \\ 
\end{minipage}



\vskip 30 pt
\noindent
{\bf Step 4: Repeat the procedure to locate the saddle point (SP).  } \\
\vskip 5 pt
\begin{minipage}{6in}
\leftskip1.5cm
Unless you are lucky or using a very large number of replicas, none of the replicas
generally lie sufficeintly close to the saddle point. You can repeat the procedure
to refine a segment of the MEP to get the SP if you are interested in the
energy barrier as well as reaction path. Convergence can be achived 
much quicker than the first run.
However, pay special attention to replicas
near the initial and final configurations because they 
may not be exactly on the MEP due to the finite tangent approximation.
Alternatively, one may use a simple procedure like Activation-Relxation technique (ART)  
[Phys. Rev. Lett. 77, 4358, (1996) ] to locate the saddle point from the replica
close to the saddle point. 
\end{minipage}


\newpage
\vskip 50 pt
\noindent
{\Large \bf Practical aspects and considerations }
\vskip 10 pt
\noindent
{\bf Step 1: Minima } \\
\vskip 5 pt
\begin{minipage}{6in}
\leftskip1.5cm
For exotic systems, finding relevant (good) minima can be challenging.
Using MD (cp option) or ``metadynamics" (implemented yet ?) is helpful 
in exploring the PES for these cases. Locating good mimima is very important!
\end{minipage}

\vskip 30 pt
\noindent
{\bf Step 2: Number of replicas } \\
\vskip 5 pt
\begin{minipage}{6in}
\leftskip1.5cm
It is {\bf crucial} to have a reasonable number of replicas to represent the string.
If the number is too small, the string will not converge and most likey
to result in Lagrange multiplier convergence failures.
Approximation to the path (tangent) is as good as the number of replicas that the string consists of.
It is a misconception that using a less number of replicas
leads to finding the MEP computationally faster. If you use a large number of replicas,
the changes in the tangential direction is small at each step and results in
more smooth convergence (therefore often faster). 
It is a good idea to check if the arclength that each replica
covers is reasonable. I typically use 10$\sim$20 replicas for most chemical reactions.
Remember that the scaling of computational cost is essentially linear with the number
of replicas.
\end{minipage}

\vskip 30 pt
{\bf Step 4: This process can be tricky } \\
\vskip 5 pt
\begin{minipage}{6in}
\leftskip1.5cm
To locate the saddle point, you restart the SM calc. for a 
segment of the original MEP located. Notice that these initial and final
points are no longer minima in the PES. 
If these points do not lie on the MEP sufficiently close  
(due to a finite number of replicas used), the string optimization
will eventaully fail at the ends of the string for the new SM calculation
In other words, the string can not converge to the MEP at the two end points.
Although it is often not a problem in practice because most
parts of the string converges to the MEP thus locating the SP, it can 
be annoying when the program terminates abruptly.
\end{minipage}


\newpage
\noindent
{\large \bf Common mistakes}
\vskip 20 pt
\noindent
1. Atom indexes are not properly assigned for the initial and final states. It is always a good idea to visualize the path.\\
\vskip 5pt
\noindent
2. Atoms are going away from the supercell, and coming back from the other side (due to PBC). \\
\vskip 5pt
\noindent
3. When studying molecules, rotational and translational degrees of freedom may ``stretch'' the string incoveniently if
the initial and final states are not set appropriately. This does not affect the any physical properties, but it is undesirable
since each replica must cover more of the path if stretched. \\
\vskip 5pt

\vskip 40 pt
\noindent
{\large \bf Trouble shootings}
\vskip 20 pt
\noindent
1. Hard to keep the wavefunction of the replicas near the TS, close to the BO surface for some reactions. \\
\vskip 7 pt
\noindent
It is because sometimes that the energy gap (HOMO-LUMO) could be extremely small at the TS. If this is the case,
use the thermostats (e.g. Nose) on electronic degrees of freedom to keep them cooled down.   \\
\vskip 20 pt

\noindent
2. Lagrange multipliers do not converge.  \\
\vskip 7pt
\noindent
A simple cure is to increase the number of
replicas, or use a smaller time step. By the numerical construction, the convergence
can be achieved for a reasonable change in tangential direction and a sufficiently small time step (usually
the lagrange mutilplier for the WF orthogonalization is the limiting one). \\   
\vskip 20 pt

\noindent
3. The code does not work. \\
\vskip 7pt
\noindent
Debug it.


\newpage
\noindent
{\large \bf Variables relevant for SMD option}
\vskip 10 pt

INTEGER :: num\_of\_images \\
\- \hspace{40pt} number of replicas \\ \vskip 10 pt
 
LOGICAL :: smd\_smcp = .TRUE. \\
\- \hspace{40pt} for regular elec. min. \\ \vskip 10 pt

LOGICAL :: smd\_smopt = .FALSE. \\
\- \hspace{40pt} for the minimization of only initial and final state. \\ \vskip 10 pt

LOGICAL :: smd\_smlm = .FALSE. \\
\- \hspace{40pt} for string mehod optimization. \\ \vskip 20 pt  


LOGICAL :: smd\_linr = .FALSE. \\
\- \hspace{40pt} linear interpolation is used to obtain  \\
\- \hspace{40pt} the initial path from the two end points. \\ \vskip 10 pt

LOGICAL :: smd\_polm = .FALSE. \\
\- \hspace{40pt} polynomial interpolation (plus grid distribution) \\
\- \hspace{40pt} to obtain the initial path. \\ \vskip 10pt

INTEGER :: smd\_kwnp = 2 \\
\- \hspace{40pt} number of known points for polm. interp. \\ \vskip 10pt

INTEGER :: smd\_codf = 50, smd\_forf = 50, smd\_smwf = 1 \\
\- \hspace{40pt} frequency of writing coord, force, replica, files \\ \vskip 20pt


INTEGER :: smd\_lmfreq = 1 \\
\- \hspace{40pt} frequencey of computing the Lagrange Multiplier  \\
\- \hspace{40pt} to satisfy the constraint on the string. \\ \vskip 10pt

REAL (KIND=DP) :: smd\_tol = 0.0001 \\*
\- \hspace{40pt} tolerance in the constraints in units of  \\*
\- \hspace{40pt} [$\alpha$(k)-$\alpha$(k-1)] - 1/SMP \\*
\- \hspace{40pt} where $\alpha$ is normalized arclength for replica k, \\*
\- \hspace{40pt} and SMP is the num\_of\_image-1  \\ \vskip 10pt

INTEGER :: smd\_maxlm = 10 \\
\- \hspace{40pt} maximum number of iterations for calculating 
Lag. mul. \\ \vskip 20 pt

REAL (KIND=DP) :: smd\_ene\_ini = 1.d0 \\
\- \hspace{40pt} Potential energy of the intital state. \\ \vskip 10 pt

REAL (KIND=DP) :: smd\_ene\_fin = 1.d0 \\
\- \hspace{40pt} Potential energy of the final state. \\ \vskip 10 pt


\vskip 40 pt
\noindent
{\large \bf Further Modification and Program architecture}
\vskip 20 pt
\noindent
In the current quantum-ESPRESSO implementation, SMD option is invoked by a separate driver routine (smcp.f90/smdmain) 
and not from the one (cpr.f90/cprmain) for the regular CP calculation.
This subroutine smdmain calls all the necessary CP subroutines as exactly in the same way as in the subroutine cprmain.
Each replica is defined using a TYPE definition, holding a set of variables common to all the replicas.
In the q-ESPRESSO implementation, parallelization over replicas (2nd level of paralellization) is NOT implemented so far, 
in order to be consistent with other parts of the package for future internal integrations among the codes. 
A detailed description to compute the free energy along the MEP (as well as any other paths) can be
found in JCP 122, 114104 (2005).     

\end{document}
