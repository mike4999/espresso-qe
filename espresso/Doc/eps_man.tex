%
% Copyright (C) 2006 Andrea Benassi
% This file is distributed under the terms of the
% GNU General Public License. See the file `License'
% in the root directory of the present distribution,
% or http://www.gnu.org/copyleft/gpl.txt .
%
\documentclass[twocolumn]{article}
\usepackage[english]{babel}
\usepackage[dvips]{epsfig}
\usepackage[centertags,intlimits]{amsmath}
\usepackage{amssymb}
\usepackage{verbatim}
\begin{document}
\begin{titlepage}
\Huge
\begin{center}
$PW_{SCF}$'s epsilon.x user's manual\\[1.5cm]
\normalsize
\textbf{Manual Autor:}
\emph{Andrea Benassi}$^{1,2}$\\[0.3cm]
\textbf{Code Developers:}
\emph{Andrea Benassi$^{1,2}$, Andrea Ferretti$^{1,2}$, Carlo Cavazzoni$^{2,3}$}\\[1cm]
$^{1}$ \emph{Physics Department, Universit\'a degli Studi di Modena e Reggio Emilia,} www.fisica.unimore.it\\
$^{2}$ \emph{INFM/S$^{3}$ (Nanostructure and Biosystem at Surfaces),} www.s3.infm.it\\
$^{3}$ \emph{High Performace Computing Department, CINECA Consorzio Interuniversitario,} www.cineca.it\\
\end{center}
\end{titlepage}
\newpage
\section{Introduction}
Epsilon.x is a post processing code of $PW_{SCF}$. Starting from a self consistent field or a non-self consistent field calculation,
epsilon.x provides the real and immaginary parts of the dielectric tensor or the joint density of states, it works both in serial and
parallel mode. As all the others post processing codes, epsilon.x must run with the same number of
processors of the previews parallel PW runs, it can run with a single processor if and only if the previews parallel PW runs where
performed with the
variable WF\_COLLECT=.TRUE..\\
Epsilon.x doesn't support the reduction of the k-points grid into the unreducible Brillouin zone, so the previews PW runs must be 
performed with a uniform k-points grid and all k-points weights must be equal to each other, i.g. in the k-points card the k-points
coordinates must be given manualy in \emph{crystal} or \emph{alat} or \emph{bohr}, but not with the \emph{automatic} option. Also the 
auto-symmetrization of k-points grid can produce a non uniform distribution of k-points weights, in order to avoid this
PW's behavior the variable NOSYM must be set .TRUE. to disable auto-symmetrizzation.
\section{Input file}
When executed epsilon.x reads an input file from standard input, this file contains two ForTran namelists:
\begin{verbatim}
 &inputpp
    outdir='/scratch/bulk/'
    prefix='bulk'
    calculation='eps' 
 /
 &energy_grid 
    smeartype='gauss'
    smear=0.15d0
    wmax=30.0d0
    nw=1000
    shift=0.0d0
 / 
\end{verbatim} 
the first two character are the location and names of the output files from the previews PW runs. \emph{calculation}
select the kind of calculation to be performed by epsilon.x, actually the following calculation are implemented:
\begin{itemize}
\item  \emph{eps}: dielectric tensor calculation, in addition to the standard output the code produces the four files
\emph{epsr.dat}, \emph{epsi.dat}, \emph{eels.dat} and \emph{ieps.dat}. The first two contain the real and immaginary 
parts of the dielectric
tensor diagonal components $\epsilon_{1_{\alpha,\alpha}}(\omega)$ e $\epsilon_{2_{\alpha,\alpha}}(\omega)$, 
as a function of frequency (in eV). The third file contains the electron energy loss spectrum calculated from the diagonal 
elements of dielectrc tensor and the last one contains the diagonal components of 
dielectric tensor calculated on the immaginary axe of frequency (via London transformation) 
$\epsilon_{\alpha,\alpha}(i\omega)$. 
\item  \emph{jdos}: joint density of state calculation, in addition to the standard output the code produces the file
\emph{jdos.dat}, containing the joint density of state (in eV$^{-1}$) as a function of frequency (in eV)
\end{itemize}
\emph{smeartype} select the kind of brodening for the plot of joint density of state, it can be both
\emph{gauss} or \emph{lorentz} for a gaussian or lorentzian brodening. \emph{smear} is the broadening parameter (in eV), 
it will be the gaussian or lorentzian broadening parameter in the case of joint density of state calculation or the 
Drude-Lorentz broadening parameter for the dielectric tensor calculation.
The desired functions will be calculated in a frequency interval $\big[$-\emph{wmax},\emph{wmax}$\big]$ and \emph{nw} 
is the number of points of the frequency mesh, \emph{wmax} is expected to be in eV. Finally \emph{shift} is the number 
of eV for an optional rigid shift of the immaginary part of the dielectric function. 
   
\section{Joint density of states}
The joint density of state is defined has:
\begin{displaymath}
n(\omega)=\sum_{\sigma}\sum_{n\in V}\sum_{n'\in C}\frac{\Omega}{(2\pi)^3}\int d^3\textbf{k}\delta(E_{\textbf{k},n'}-E_{\textbf{k},n}
-\hbar\omega)
\end{displaymath}
or alternatively:
\begin{equation}
n(\omega)=\sum_{n}\sum_{n'}\frac{\Omega}{(2\pi)^3}\int d^3\textbf{k}\delta(E_{\textbf{k},n'}-E_{\textbf{k},n}
-\hbar\omega)...
\label{imp2}
\end{equation}
\begin{displaymath}
...f(E_{\textbf{k},n})[2-f(E_{\textbf{k},n'})]/2
\end{displaymath}
or finally:
\begin{equation}
n(\omega)=\sum_{n\in V}\sum_{n'\in C}\frac{\Omega}{(2\pi)^3}\int d^3\textbf{k}\delta(E_{\textbf{k},n'}-E_{\textbf{k},n}
-\hbar\omega)...
\label{imp}
\end{equation}
\begin{displaymath}
...[f(E_{\textbf{k},n})-f(E_{\textbf{k},n'})]
\end{displaymath}
were $\sigma$ is the spin component, $\Omega$ is the volume of the lattice cell, $n$ and $n'$ belong respectively to the 
valence and conduction bands, 
$E_{\textbf{k},n}$ are the eighenvalues of the hamiltonian and $f(E_{\textbf{k},n})$ is the Fermi distribution function 
that account for the occupation of the bands. In the last two notation the sum over spin values is included into
Fermi function whose normalizzation is two instead of one.   
The Dirac Delta function it's numerically implemented by means of Lorentzian 
or Gaussian functions normalized to one:
\begin{equation}
L(\omega)=\frac{\Gamma}{\pi\big[(E_{\textbf{k},n'}-E_{\textbf{k},n}-\hbar\omega)^2+\Gamma^2\big]}
\label{lor}
\end{equation}
\begin{equation}
G(\omega)=\frac{1}{\Gamma\sqrt{\pi}}e^{(E_{\textbf{k},n'}-E_{\textbf{k},n}-\hbar\omega)^2/\Gamma^2}
\label{gau}
\end{equation}
$\Gamma$ is the brodening parameter from the input file. The implemented formula is obtained sobstituting the 
Dirac Delta function in (\ref{imp}) by (\ref{lor}) or (\ref{gau}) and sobstituting $\frac{\Omega}{(2\pi)^3}\int 
d^3\textbf{k}$ by a simple sun over k-points.\\
Integrating analitically (\ref{imp}) one obtains: 
\begin{eqnarray}
\sum_{\textbf{k}}\sum_{n}\sum_{n'}[f(E_{\textbf{k},n})-f(E_{\textbf{k},n'})]
\end{eqnarray}
so a division by this quantity is needed to renormalize to one the joint density of state, the standard output file 
contains a convergence check on this renormalizzazion. Note that in the case of 
joint density of state the two kinds of brodening (\ref{lor}) and (\ref{gau}) are exactly equivalent.

\section{Dielectric tensor}
The immaginary part of the dielectric tensor $\epsilon_{2_{\alpha,\beta}}(\omega)$ can be viewed as a response function 
that comes from a perturbation theory with adiabatic turning on:
\begin{displaymath}
\epsilon_{\alpha,\beta}(\omega)=1+\frac{4 \pi e^2}{\Omega N_{\textbf{k}} m^2}\sum_{n\in V}\sum_{n'\in C}\sum_{\textbf{k}}
\frac{|\hat{\textbf{p}}_{\alpha,\beta}|^2}{(E_{\textbf{k},n'}-E_{\textbf{k},n})^2}...
\end{displaymath}
\begin{displaymath}
...\Bigg\{\frac{f(E_{\textbf{k},n})}{E_{\textbf{k},n'}-E_{\textbf{k},n}+\hbar\omega+i\hbar\Gamma}+...
\end{displaymath}
\begin{equation}
...\frac{f(E_{\textbf{k},n})}{E_{\textbf{k},n'}-E_{\textbf{k},n}-\hbar\omega-i\hbar\Gamma}\Bigg\}
\end{equation}
where $\Gamma$ is the adiabatic parameter and, for the total energy conservation it must tend to zeto. This is the way in 
which the Dirac Delta function appears and this means that every excited state has an infinite lifetime, i.g. is stationary. 
\begin{displaymath}
\epsilon_{2_{\alpha,\beta}}(\omega)=\frac{4 \pi e^2}{\Omega N_{\textbf{k}} m^2}\sum_{n\in V}\sum_{n'\in C}\sum_{\textbf{k}}
\frac{|\hat{\textbf{p}}_{\alpha,\beta}|^2 f(E_{\textbf{k},n})}{(E_{\textbf{k},n'}-E_{\textbf{k},n})^2}...
\end{displaymath}
\begin{equation}
...\bigg[\delta(E_{\textbf{k},n'}-E_{\textbf{k},n}+\hbar\omega)+\delta(E_{\textbf{k},n'}-
E_{\textbf{k},n}-\hbar\omega)\bigg]
\end{equation}
This situation is unphysical because the interaction with 
electromagnetic field (even in the absence of photons, i.g. spontaneous emission) gives an intrinsic broadening to all exited 
states, the lifetime is finite and $\Gamma$ must be greater than zero. In the limit of small but non vanishing $\Gamma$ 
the dielectric tensor turns into the Drude-Lorentz one: 
\begin{displaymath}
\epsilon_{2_{\alpha,\beta}}(\omega)=\frac{8 \pi e^2}{\Omega N_{\textbf{k}} m^2}\sum_{n\in V}\sum_{n'\in C}\sum_{\textbf{k}}
\frac{|\hat{\textbf{p}}_{\alpha,\beta}|^2}{E_{\textbf{k},n'}-E_{\textbf{k},n}}...
\end{displaymath}
\begin{equation}
...\frac{\Gamma \omega f(E_{\textbf{k},n})}{\big[(E_{\textbf{k},n'}-E_{\textbf{k},n})^2-\hbar\omega^2\big]^2+\Gamma^2\hbar\omega^2}
\end{equation}
while the real part comes from the Krames-Kronig transformation:
\begin{equation}
\epsilon_{1_{\alpha,\beta}}(\omega)=1+\frac{2}{\pi}\int_{0}^{\infty}\frac{\omega' \epsilon_{2_{\alpha,\beta}}(\omega')}
{\omega'^{2}-\omega^{2}}d\omega'
\end{equation} 
\begin{displaymath}
\epsilon_{1_{\alpha,\beta}}(\omega)=1+\frac{8 \pi e^2}{\Omega N_{\textbf{k}} m^2}\sum_{n\in V}\sum_{n'\in C}\sum_{\textbf{k}}
\frac{|\hat{\textbf{p}}_{\alpha,\beta}|^2}{E_{\textbf{k},n'}-E_{\textbf{k},n}}...
\end{displaymath}
\begin{equation}
...\frac{\big[(\omega_{\textbf{k},n'}-\omega_{\textbf{k},n})^2-\omega^2\big]f(E_{\textbf{k},n})}{\big[(E_{\textbf{k},n'}-E_{\textbf{k},n})^2
-\hbar\omega^2\big]^2+\Gamma^2\hbar\omega^2}
\end{equation}
The squared matrix elements are defined as follow:
\begin{equation}
\hat{\textbf{p}}_{\alpha,\beta}=\langle\psi_{\textbf{k},n'}\vert\hat{\textbf{p}}_{\alpha}\vert \psi_{\textbf{k},n}\rangle
\langle\psi_{\textbf{k},n}\vert\hat{\textbf{p}}_{\beta}^{\dagger}\vert\psi_{\textbf{k},n'}\rangle
\label{nos}
\end{equation}
\begin{equation}
\propto \psi_{\textbf{k},n'}^{\star}(\textbf{r})\frac{d}{d x_{\alpha}}\psi_{\textbf{k},n}(\textbf{r})
\psi_{\textbf{k},n}^{\star}(\textbf{r})\frac{d}{d x_{\beta}}\psi_{\textbf{k},n'}(\textbf{r})
\end{equation}
the double index reveals the tensorial nature of $\epsilon_{2}(\omega)$, while $\vert \psi_{\textbf{k},n}\rangle$ is a
single particle Bloch function obtained by the PW's DFT calculation. 
The calculation of the off-diagonal elements of 
the tensor $\epsilon_{2_{\alpha,\beta}}(\omega)$ is not implemented but it is trivial to obtain from the preview definition. 
In all the cases illustrated above the non-local contriution due to the pseudopotential is neglected, actually the 
correction to the matrix element that take into account the non-local part of the hamiltonian it's not implemented.
From the previews definition of the immaginary part of the dielettric function it is easy to see that even the local-field 
contributions are not implemented.\\
PW works on a plane wave set so the Bloch functions of the matrix element (\ref{nos}) are decomposed as follow:
\begin{equation}
\vert \psi_{\textbf{k},n}\rangle=\frac{1}{\sqrt{V}}\sum_{\textbf{G}}a_{n,\textbf{k},\textbf{G}}
e^{i(\textbf{k}+\textbf{G})\cdot\textbf{r}}
\end{equation}
and consequently:
\begin{equation}
\hat{\textbf{p}}_{\alpha,\alpha}=\vert\sum_{\textbf{G}}a^{\star}_{n,\textbf{k},\textbf{G}}a_{n',\textbf{k},\textbf{G}}
(k_{\alpha}+G_{\alpha})\vert^{2}
\end{equation}
defined in this way the matrix element accounts only for interband transitions, i.e. vertical transition in which the 
electron momentum $\textbf{k}$ is conserved (optical approximation). In standard optics the intraband transitions give a 
neglectable contribution due to the very low momentum transfered by the incomming/outcomming photon.\\
Operating a London transformation upon $\epsilon_{2_{\alpha,\beta}}(\omega)$, it's possible to obtain the whole dielectric
tensor calculated on the immaginary frequency axe $\epsilon_{\alpha,\beta}(i\omega)$. 
\begin{equation}
\epsilon_{\alpha,\beta}(i\omega)=1+\frac{2}{\pi}\int_{0}^{\infty}\frac{\omega' \epsilon_{2_{\alpha,\beta}}(\omega')}
{\omega'^{2}+\omega^{2}}d\omega'
\end{equation}
The LOSS spectrum is proportional to the immaginary of the inverse dielectric tensor, that is:
\begin{equation}
Imm\Bigg\{\frac{1}{\epsilon_{\alpha,\beta}(\omega)}\Bigg\}=
\frac{\epsilon_{2_{\alpha,\beta}}(\omega)}{\epsilon_{1_{\alpha,\beta}}^{2}(\omega)+
\epsilon_{2_{\alpha,\beta}}^{2}(\omega)}
\end{equation}
this quantity provides a useful check of the dielectric tensor calculation becasue it reaches its maximum at the bulk plasmon
frequency $\Omega_{p}$, where the real and immaginary parts cross their paths at higher frequency. The same quantity (in eV)
is numerically evaluated using the following sum rule:
\begin{equation}
\int_{0}^{\infty}\omega\epsilon_{2_{\alpha,\beta}}(\omega)d\omega=\frac{\pi}{2}\Omega_{p}
\end{equation}  
The result of this calculation is printed in the standard output file.
\end{document}

