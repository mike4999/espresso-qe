\documentclass[12pt,a4paper]{article}
 \def\version{2.1}

 \usepackage{epsfig}
 %\usepackage{html}
  \def\htmladdnormallink#1#2{#1}

\begin{document}

\section{FPMD/CP Manual}

This is a very short manual about FPMD and CP codes, 
and it is intended to explain how to perform basic Car-Parrinello (CP)
simulations.

It is important to understand that a CP simulation is a sequence of 
different runs, some of them used to "prepare" the initial state
of the system, and other performed to collect statistics,
or to modify the state of the system itself, i.e. modify the temperature
or the pressure.

To prepare and run a CP simulation you should:

\begin{enumerate}
  \item
    define the system:
    \begin{enumerate}
      \item atomic positions
      \item system cell
      \item pseudopotentials
      \item number of electrons and bands
      \item cut-offs 
      \item FFT grids (CP code only)
    \end{enumerate}

  \item
    The first run, when starting from scratch, is always an electronic
    minimization, with fixed ions and cell, to bring the electronic
    system on the ground state (GS) relative to the starting atomic
    configuration.
    Example of input file (Benzene Molecule):
\begin{verbatim}
 &control
    title = ' Benzene Molecule ',
    calculation = 'cp',
    restart_mode = 'from_scratch',
    ndr = 51,
    ndw = 51,
    nstep  = 100,
    iprint = 10, 
    isave  = 100,
    tstress = .TRUE.,
    tprnfor = .TRUE.,
    dt    = 5.0d0,
    etot_conv_thr = 1.d-9,
    ekin_conv_thr = 1.d-4,
    prefix = 'c6h6'
    pseudo_dir='/scratch/acv0/benzene/',
    outdir='/scratch/acv0/benzene/Out/'
 /
 &system
    ibrav = 14, 
    celldm(1) = 16.0, 
    celldm(2) = 1.0, 
    celldm(3) = 0.5, 
    celldm(4) = 0.0, 
    celldm(5) = 0.0, 
    celldm(6) = 0.0, 
    nat  = 12,
    ntyp = 2,
    nbnd = 15,
    nelec = 30,
    ecutwfc = 40.0,
    nr1b= 10, nr2b = 10, nr3b = 10,
    xc_type = 'BLYP'
 /
 &electrons
    emass = 400.d0,
    emass_cutoff = 2.5d0,
    electron_dynamics = 'sd',
 /
 &ions
    ion_dynamics = 'none',
 /
 &cell
    cell_dynamics = 'none',
    press = 0.0d0,
 /
ATOMIC_SPECIES
 C 12.0d0 c_blyp_gia.pp
 H 1.00d0 h.ps
ATOMIC_POSITIONS (bohr)
   C     2.6  0.0 0.0
   C     1.3 -1.3 0.0
   C    -1.3 -1.3 0.0
   C    -2.6  0.0 0.0
   C    -1.3  1.3 0.0
   C     1.3  1.3 0.0
   H     4.4  0.0 0.0
   H     2.2 -2.2 0.0
   H    -2.2 -2.2 0.0
   H    -4.4  0.0 0.0
   H    -2.2  2.2 0.0
   H     2.2  2.2 0.0
\end{verbatim}

    You can find the description of the input variables in files
    \texttt{INPUT.FPMD}, \texttt{INPUT.HOWTO} and \texttt{INPUT}.

  \item
    Sometimes a single run is not enough to reach the GS.
    In this case, you need to re-run the electronic minimization
    stage.
    Use the input of the first run, changing \texttt{restart\_mode =
    'from\_scratch'} to \texttt{restart\_mode = 'restart'}.

    Important: unless you are already experienced with the system you
    are studying or with the code internals, usually you need to tune
    some input parameters, like \texttt{emass}, \texttt{dt}, and
    cut-offs.
    For this purpose, a few trial runs could be useful: you can
    perform short minimizations (say, 10 steps) changing and adjusting
    these parameters to your need.

    You could specify the degree of convergence with these two
    thresholds:

    \texttt{etot\_conv\_thr}: total energy difference between two
    consecutive steps

    \texttt{ekin\_conv\_thr}: value of the fictitious kinetic energy
    of the electrons

    Usually we consider the system on the GS when
    \texttt{ekin\_conv\_thr}${} < \sim 10^{-5}$.
    You could check the value of the fictitious kinetic energy on the 
    standard output (column EKINC).

    Different strategies are available to minimize electrons, but the
    most used ones are:
    \begin{itemize}
      \item
        steepest descent:
\begin{verbatim}
  electron_dynamics = 'sd'
\end{verbatim}
      \item
         damped dynamics:
\begin{verbatim}
  electron_dynamics = 'damp', 
  electron_damping = 0.1,
\end{verbatim}
         See input description to compute damping factor, usually the
         value is between 0.1 and 0.5.
    \end{itemize}

  \item
    Once your system is in the GS, depending on how you have prepared
    the starting atomic configuration, you should do several things:
    \begin{itemize}
      \item
        if you have set the atomic positions ``by hand'' and/or from a
        classical code, check the forces on atoms, and if they are
        large ($\sim 0.1 - 1.0$ atomic units), you should perform an
        ionic minimization, otherwise the sistem could break-up during
        the dynamics.
      \item
        if you have taken the positions from a previous run or a
        previous ab-initio simulation, check the forces, and if they
        are too small ($\sim 10^{-4}$ atomic units), this means that
        atoms are already in equilibrium positions and, even if left
        free, they will not move.
        Then you need to randomize positions a little bit. see below.
    \end{itemize}

  \item
    Minimize ionic positions.

    As we pointed out in 4) if the interatomic forces are too high,
    the system could "explode" if we switch on the ionic dynamics.
    To avoid that we need to relax the system. 

    Again there are different strategies to relax the system, but the
    most used are again steepest descent or damped dynamics for ions
    and electrons.
    You could also mix electronic and ionic minimization scheme
    freely, i.e. ions in steepest and electron in damping or vice
    versa.
   
    \begin{enumerate}
      \item
        suppose we want to perform a steepest for ions.
        Then we should specify the following section for ions:
\begin{verbatim}
 &ions
    ion_dynamics = 'sd',
 /
\end{verbatim}
        Change also the ionic masses to accelerate the minimization:
\begin{verbatim}
ATOMIC_SPECIES
 C 2.0d0 c_blyp_gia.pp
 H 2.00d0 h.ps
\end{verbatim}
        while leaving unchanged other input parameters.   

        Note that if the forces are really high ($> 1.0$ atomic
        units), you should always use stepest descent for the first
        relaxation steps ($\sim 100$).
   
      \item
        as the system approaches the equilibrium positions, the
        steepest descent scheme slows down, so is better to switch to
        damped dynamics:
\begin{verbatim}
 &ions
    ion_dynamics = 'damp',
    ion_damping = 0.2,
    ion_velocities = 'zero',
 /
\end{verbatim}
        A value of \texttt{ion\_damping} between 0.05 and 0.5 is
        usually used for many systems.
        It is also better to specify to restart with zero ionic and
        electronic velocities, since we have changed the masses.
        Change further the ionic masses to accelerate the
        minimization:
\begin{verbatim}
ATOMIC_SPECIES
 C 0.1d0 c_blyp_gia.pp
 H 0.1d0 h.ps
\end{verbatim}

      \item
        when the system is really close to the equilibrium, the damped
        dynamics slow down too, especially because, since we are
        moving electron and ions together, the ionic forces are not
        properly correct, then it is often better to perform a ionic
        step every $N$ electronic steps, or to move ions only when
        electron are in their GS (within the chosen threshold).

        This can be specified adding, in the ionic section, the
        \texttt{ion\_nstepe} parameter, then the ionic input section
        become as follows:
\begin{verbatim}
 &ions
    ion_dynamics = 'damp',
    ion_damping = 0.2,
    ion_velocities = 'zero',
    ion_nstepe = 10,
 /
\end{verbatim}
        Then we specify in the control input section:
\begin{verbatim}
    etot_conv_thr = 1.d-6,
    ekin_conv_thr = 1.d-5,
    forc_conv_thr = 1.d-3
\end{verbatim}
        As a result, the code checks every 10 electronic steps whether
        the electronic system satisfies the two thresholds
        \texttt{etot\_conv\_thr}, \texttt{ekin\_conv\_thr}: if it
        does, the ions are advanced by one step.
        The process thus continues until the forces become smaller
        than \texttt{forc\_conv\_thr}.

        Note that to fully relax the system you need many run, and
        different strategies, that you shold mix and change in order
        to speed-up the convergence.
        The process is not automatic, but is strongly based on
        experience, and trial and error.
     
        Remember also that the convergence to the equilibrium
        positions depends on the energy threshold for the electronic
        GS, in fact correct forces (required to move ions toward the
        minimum) are obtained only when electrons are in their GS.
        Then a small threshold on forces could not be satisfied, if
        you do not require an even smaller threshold on total energy.
    \end{enumerate}

  \item
    randomization of positions.

    If you have relaxed the system or if the starting system is
    already in the equilibrium positions, then you need to move ions
    from the equilibrium positions, otherwise they won't move in a
    dynamics simulation.
    After the randomization you should bring electrons on the GS
    again, in order to start a dynamic with the correct forces and
    with electrons in the GS.
    Then you should switch off the ionic dynamics and activate the
    randomization for each species, specifying the amplitude of the
    randomization itself.
    This could be done with the following ionic input section:
\begin{verbatim}
 &ions
    ion_dynamics = 'none',
    tranp(1) = .TRUE.,
    tranp(2) = .TRUE.,
    amprp(1) = 0.01
    amprp(2) = 0.01
 /
\end{verbatim}
    In this way a random displacement (of max 0.01 a.u.) is added to
    atoms of specie 1 and 2.
    All other input parameters could remain the same.

    Note that the difference in the total energy (\texttt{etot})
    between relaxed and randomized positions can be used to estimate
    the temperature that will be reached by the system.
    In fact, starting with zero ionic velocities, all the difference
    is potential energy, but in a dynamics simulation, the energy will
    be equipartitioned between kinetic and potential, then to estimate
    the temperature take the difference in energy (de), convert it in
    Kelvins, divide for the number of atoms and multiply by 2/3.

    Randomization could be useful also while we are relaxing the
    system, especially when we suspect that the ions are in a local
    minimum or in an energy plateau.
   
  \item
    Start the Car-Parrinello dynamics.

    At this point after having minimized the electrons, and with ions
    displaced from their equilibrium positions, we are ready to start
    a CP dynamics.
    We need to specify \texttt{'verlet'} both in ionic and electronic
    dynamics.
    The threshold in control input section will be ignored, like any
    parameter related to minimization strategy.
    The first time we perform a CP run after a minimization, it is
    always better to put velocities equal to zero, unless we have
    velocities, from a previous simulation, to specify in the input
    file.
    Restore the proper masses for the ions.
    In this way we will sample the microcanonical ensemble.
    The input section changes as follow:
\begin{verbatim}
 &electrons
    emass = 400.d0,
    emass_cutoff = 2.5d0,
    electron_dynamics = 'verlet',
    electron_velocities = 'zero',
 /
 &ions
    ion_dynamics = 'verlet',
    ion_velocities = 'zero',
 /
ATOMIC_SPECIES
C 12.0d0 c_blyp_gia.pp
H 1.00d0 h.ps
\end{verbatim}
    If you want to specify the initial velocities for ions, you have
    to set \texttt{ion\_velocities = 'from\_input'}, and add the
    \texttt{IONIC\_VELOCITIES}\\ card, with the list of velocities in
    atomic units.

    IMPORTANT: in restarting the dynamics after the first CP run,
    remember to remove or comment the velocities parameters:
\begin{verbatim}
 &electrons
    emass = 400.d0,
    emass_cutoff = 2.5d0,
    electron_dynamics = 'verlet',
    ! electron_velocities = 'zero',
 /
 &ions
    ion_dynamics = 'verlet',
    ! ion_velocities = 'zero',
 /
\end{verbatim}
    otherwise you will quench the system interrupting the sampling of
    the microcanonical ensemble.

  \item
    Changing the temperature of the system.

    It is possible to change the temperature of the system or to
    sample the canonical ensemble fixing the average temperature, this
    is done using the Nos\`e thermostat.
    To activate this thermostat for ions you have to specify in the
    ions input section:
\begin{verbatim}
 &ions
    ion_dynamics = 'verlet',
    ion_temperature = 'nose',
    fnosep = 60.0,
    tempw  = 300.0,
    ! ion_velocities = 'zero',
 /
\end{verbatim}
    where \texttt{fnosep} is the frequency of the thermostat in THz,
    this should be chosen to be comparable with the center of the
    vibrational spectrum of the system, in order to excite as many
    vibrational modes as possible.
    \texttt{tempw} is the desired average temperature in Kelvin.

    It is possible to specify also the thermostat for the electrons,
    this is usually activated in metal or in system where we have a
    transfer of energy between ionic and electronic degrees of
    freedom.
\end{enumerate}

\end{document}
