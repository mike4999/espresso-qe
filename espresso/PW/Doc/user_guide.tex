\documentclass[12pt,a4paper]{article}
\def\version{5.0}
\def\PWscf{\texttt{PWscf}}
\def\qe{{\sc Quantum ESPRESSO}}

\usepackage{html}

% BEWARE: don't revert from graphicx for epsfig, because latex2html
% doesn't handle epsfig commands !!!
\usepackage{graphicx}

\textwidth = 17cm
\textheight = 24cm
\topmargin =-1 cm
\oddsidemargin = 0 cm

\def\pwx{\texttt{pw.x}}
\def\cpx{\texttt{cp.x}}
\def\phx{\texttt{ph.x}}
\def\nebx{\texttt{neb.x}}
\def\configure{\texttt{configure}}

\def\PHonon{\texttt{PHonon}}
\def\CP{\texttt{CP}}
\def\PostProc{\texttt{PostProc}}
\def\make{\texttt{make}}

\begin{document} 
\author{}
\date{}

\def\qeImage{../../Doc/quantum_espresso.pdf}
\def\democritosImage{../../Doc/democritos.pdf}

\begin{htmlonly}
\def\qeImage{../../Doc/quantum_espresso.png}
\def\democritosImage{../../Doc/democritos.png}
\end{htmlonly}

\title{
  \includegraphics[width=5cm]{\qeImage} \hskip 2cm
  \includegraphics[width=6cm]{\democritosImage}\\
  \vskip 1cm
  % title
  \Huge User's Guide for \PWscf\smallskip
  \Large (version \version)
}
%\endhtmlonly

%\latexonly
%\title{
% \epsfig{figure=quantum_espresso.png,width=5cm}\hskip 2cm
% \epsfig{figure=democritos.png,width=6cm}\vskip 1cm
%  % title
%  \Huge User's Guide for \qe \smallskip
%  \Large (version \version)
%}
%\endlatexonly

\maketitle

\tableofcontents

\section{Introduction}

This guide covers the usage of the \PWscf\ 
(Plane-Wave Self-Consistent Field) package,
a core component of the \qe\ distribution.
Further documentation, beyond what is provided 
in this guide, can be found in the directory
\texttt{PW/Doc/}, containing a copy of this guide.

This guide assumes that you know the physics 
that \PWscf\ describes and the methods it implements.
It also assumes  that you have already installed,
or know how to install, \qe. If not, please read
the general User's Guide for \qe, found in 
directory \texttt{Doc/} two levels above the 
one containing this guide; or consult the web site:\\
\texttt{http://www.quantum-espresso.org}.

People who want to modify or contribute to 
\PWscf\ should read the Developer Manual: 
\texttt{Doc/developer\_man.pdf}.

\subsection{What can \PWscf\ do}

\PWscf\ performs many different kinds of
self-consistent calculations of electronic-structure
properties within
Density-Functional Theory (DFT), using a Plane-Wave (PW) basis set and pseudopotentials (PP).
In particular:
\begin{itemize}
  \item ground-state energy and one-electron (Kohn-Sham) orbitals;
  \item atomic forces, stresses, and structural optimization;
  \item molecular dynamics on the ground-state Born-Oppenheimer surface,  also with variable cell;
  \item macroscopic polarization and finite electric fields via 
  the modern theory of polarization (Berry Phases).
  \item the modern theory of polarization (Berry Phases).
  \item free-energy surface calculation at fixed cell through meta-dynamics, if patched with PLUMED.
\end{itemize}
All of the above works for both insulators and metals, 
in any crystal structure, for many exchange-correlation (XC) functionals
(including spin polarization, DFT+U, nonlocal VdW functional,
hybrid functionals), for
norm-conserving (Hamann-Schluter-Chiang) PPs (NCPPs) in 
separable form or Ultrasoft (Vanderbilt) PPs (USPPs)
or Projector Augmented Waves (PAW) method.
Non-collinear magnetism and spin-orbit interactions 
are also implemented.  An implementation of finite electric 
fields with a sawtooth potential in a supercell is also available.

\subsection{People}

The \PWscf\ package (which included \PHonon\ 
and \PostProc\ in earlier releases)
was originally developed by Stefano Baroni, Stefano
de Gironcoli, Andrea Dal Corso (SISSA), Paolo Giannozzi (Univ. Udine), and many others.
We quote in particular:
\begin{itemize}
  \item Matteo Cococcioni (Univ. Minnesota) for DFT+U implementation;
  \item David Vanderbilt's group at Rutgers for Berry's phase
  calculations;
  \item Ralph Gebauer (ICTP, Trieste) and Adriano Mosca Conte
  (SISSA, Trieste) for noncolinear magnetism;
  \item Andrea Dal Corso for spin-orbit interactions;
  \item Carlo Sbraccia (Princeton) for improvements to structural
  optimization and to many other parts;
  \item Paolo Umari (Univ. Padua) for finite electric fields;
  \item Renata Wentzcovitch and collaborators (Univ. Minnesota)
  for variable-cell molecular dynamics;
  \item Lorenzo Paulatto (Univ.Paris VI) for PAW implementation, 
  built upon previous work by Guido Fratesi (Univ.Milano Bicocca)
  and Riccardo Mazzarello (ETHZ-USI Lugano);
 \item Ismaila Dabo (INRIA, Palaiseau) for electrostatics with
 free boundary conditions;
 \item Norbert Nemec and Mike Towler (U.Cambridge) for interface with 
 \texttt{CASINO};
  \item Alexander Smogunov (CEA) for extended and non-colnear DFT+U
  implementation.
\end{itemize}

% \texttt{PWgui} was written by Anton Kokalj (IJS Ljubljana) and is 
% based on his GUIB concept (\texttt{http://www-k3.ijs.si/kokalj/guib/}).

% \texttt{iotk} (\texttt{http://www.s3.infm.it/iotk}) was written by Giovanni Bussi  (SISSA)  .

Other relevant contributions to \PWscf:
\begin{itemize}
  \item Minoru Otani (AIST), Yoshio Miura (Tohoku U.), 
  Nicephore Bonet (MIT), Nicola Marzari (Univ. Oxford), 
  Brandon Wood (LLNL), Tadashi Ogitsu (LLNL), contributed
  Effective Screening Method (PRB 73, 115407 [2006])
  \item Brian Kolb and Timo Thonhauser (Wake Forest University)
  implemented the vdW-DF and vdW-DF2 functionals, with support from
  Riccardo Sabatini and Stefano de Gironcoli (SISSA and DEMOCRITOS);
  \item Hannu-Pekka Komsa (CSEA/Lausanne) contributed
  the HSE functional;
  \item Dispersions interaction in the framework of DFT-D were
  contributed by Daniel Forrer (Padua Univ.) and Michele Pavone
  (Naples Univ. Federico II);
 \item Filippo Spiga (ICHEC) contributed the
  mixed MPI-OpenMP parallelization;
  \item The initial BlueGene porting was done by Costas Bekas and
  Alessandro Curioni (IBM Zurich).
\end{itemize}

This guide was mostly written by Paolo Giannozzi.
Mike Towler
wrote the \PWscf\ to \texttt{CASINO} subsection.

 \subsection{Terms of use}

\qe\ is free software, released under the 
GNU General Public License. See
\texttt{http://www.gnu.org/licenses/old-licenses/gpl-2.0.txt}, 
or the file License in the distribution).
    
We shall greatly appreciate if scientific work done using this code will 
contain an explicit acknowledgment and the following reference:
\begin{quote}
P. Giannozzi, S. Baroni, N. Bonini, M. Calandra, R. Car, C. Cavazzoni,
D. Ceresoli, G. L. Chiarotti, M. Cococcioni, I. Dabo, A. Dal Corso,
S. Fabris, G. Fratesi, S. de Gironcoli, R. Gebauer, U. Gerstmann,
C. Gougoussis, A. Kokalj, M. Lazzeri, L. Martin-Samos, N. Marzari,
F. Mauri, R. Mazzarello, S. Paolini, A. Pasquarello, L. Paulatto,
C. Sbraccia, S. Scandolo, G. Sclauzero, A. P. Seitsonen, A. Smogunov,
P. Umari, R. M. Wentzcovitch, J.Phys.:Condens.Matter 21, 395502 (2009),
http://arxiv.org/abs/0906.2569
\end{quote}
Note the form \qe\ for textual citations of the code.
Pseudopotentials should be cited as (for instance)
\begin{quote}
[ ] We used the pseudopotentials C.pbe-rrjkus.UPF
and O.pbe-vbc.UPF from\\
\texttt{http://www.quantum-espresso.org}.
\end{quote}

\section{Compilation}

\PWscf\ is included in the core \qe\ distribution.
Instruction on how to install it can be found in the
general documentation (User's Guide) for \qe.

Typing \texttt{make pw} from the main \qe\ directory or
\texttt{make} from the \texttt{PW/} subdirectory produces
the \pwx\ executable in \texttt{PW/src} and a link to the
\texttt{bin/} directory. In addition, several utility
programs, and related links in \texttt{bin/}, are produced 
in \texttt{PW/tools}:

\begin{itemize}
\item  \texttt{PW/tools/dist.x} calculates distances and angles between atoms in a cell,
  taking into account periodicity 
\item  \texttt{PW/tools/ev.x} fits energy-vs-volume data to an equation of state
\item  \texttt{PW/tools/kpoints.x} produces lists of k-points
\item  \texttt{PW/tools/pwi2xsf.sh},  \texttt{pwo2xsf.sh} process respectively input and output
  files (not data files!) for \pwx and produce an XSF-formatted file
  suitable for plotting with XCrySDen:
  \texttt{http://www.xcrysden.org/},  powerful crystalline and
  molecular structure visualization program.
  BEWARE: the  \texttt{pwi2xsf.sh} shell script
  requires the  \texttt{pwi2xsf.x} executables to be located somewhere in your PATH. 
\item  \texttt{PW/tools/band\_plot.x}: undocumented and possibly obsolete 
\item  \texttt{PW/tools/bs.awk},  \texttt{PW/tools/mv.awk} are scripts that process the output of \pwx\ (not
data files!). Usage: 
\begin{verbatim}
         awk -f bs.awk < my-pw-file > myfile.bs
         awk -f mv.awk < my-pw-file > myfile.mv
\end{verbatim}
The files so produced are suitable for use with  \texttt{xbs}, a very simple
X-windows utility to display molecules, available at:\\
 \texttt{http://www.ccl.net/cca/software/X-WINDOW/xbsa/README.shtml}
\item  \texttt{PW/tools/kvecs\_FS.x},  \texttt{PW/tools/bands\_FS.x}: utilities for Fermi Surface plotting
  using XCrySDen (contributed by the late Prof. Eyvaz)
\end{itemize}

\newpage\section{Using \PWscf}

Input files for \texttt{pw.x} may be either written by hand 
or produced via the \texttt{PWgui} graphical interface by Anton Kokalj, 
included in the \qe\ distribution. See \texttt{PWgui-x.y.z/INSTALL}
(where x.y.z is the version number) for more info on \texttt{PWgui}, 
or \texttt{GUI/README} if you are using SVN sources.
    
You may take the tests and examples distributed 
with \qe\ as templates for writing your own input files.
In the following, whenever we mention "Example N", we refer to those. 
Input files are those in the \texttt{results/} subdirectories, with names ending
with \texttt{.in} 
(they will appear after you have run the examples).

\subsection{Input data}

Input data 
is organized as several namelists, followed by other fields
introduced by keywords. The namelists are

\begin{tabular}{ll}
      \&CONTROL:& general variables controlling the run\\
      \&SYSTEM: &structural information on the system under investigation\\
      \&ELECTRONS: &electronic variables: self-consistency, smearing\\
      \&IONS (optional): &ionic variables: relaxation, dynamics\\
      \&CELL (optional): &variable-cell optimization or dynamics\\
\end{tabular}    \\
Optional namelist may be omitted if the calculation to be performed
does not require them. This depends on the value of variable 
\texttt{calculation}
in namelist \&CONTROL. Most variables in namelists have default values. Only
the following variables in \&SYSTEM must always be specified:

\begin{tabular}{lll}
      \texttt{ibrav} & (integer)& Bravais-lattice index\\
      \texttt{celldm} &(real, dimension 6)& crystallographic constants\\
      \texttt{nat} &(integer)& number of atoms in the unit cell\\
      \texttt{ntyp} &(integer)& number of types of atoms in the unit cell\\
      \texttt{ecutwfc} &(real)& kinetic energy cutoff (Ry) for wavefunctions.
\end{tabular}    \\
For metallic systems, you have to specify how metallicity is treated
in
variable \texttt{occupations}. If you choose \texttt{occupations='smearing'},
you have
to specify the smearing type \texttt{smearing} and the smearing width 
\texttt{degauss}. Spin-polarized systems are as a rule treated as metallic 
system, unless the total magnetization, \texttt{tot\_magnetization}
is set to a fixed value, or if occupation numbers are fixed
(\texttt{occupations='from input'} and card OCCUPATIONS).
    
Explanations for the meaning of variables \texttt{ibrav} and \texttt{celldm},
as well as on alternative ways to input structural data,
are in files \texttt{PW/Doc/INPUT\_PW.txt} and \texttt{PW/Doc/INPUT\_PW.html}. 
These files are the reference for input data and describe a large number
of other variables as well. Almost all variables have default 
values, which may or may not fit your needs.
    
After the namelists, you have several fields (``cards'')
introduced by keywords with self-explanatory names:
\begin{quote}
       ATOMIC\_SPECIES\\
       ATOMIC\_POSITIONS\\
       K\_POINTS\\
       CELL\_PARAMETERS (optional)\\
       OCCUPATIONS (optional)\\
\end{quote}
The keywords may be followed on the same line by an option. Unknown
fields are ignored. 
See the files mentioned above for details on the available ``cards''.
 
Note about k points: The k-point grid can be either automatically generated 
or manually provided as a list of k-points and a weight in the Irreducible
Brillouin Zone only of the Bravais lattice of the crystal. The code will 
generate (unless instructed not to do so: see variable \texttt{nosym}) all
required k-points
and weights if the symmetry of the system is lower than the symmetry of the
Bravais lattice. The automatic generation of k-points follows the convention
of Monkhorst and Pack.

\subsection{Data files}

The output data files are written in the directory \texttt{outdir/prefix.save}, as specified by variables
\texttt{outdir} and \texttt{prefix} 
(a string that is prepended
to all file names, whose default value is: \texttt{prefix='pwscf'}). \texttt{outdir} can be specified
as well in environment variable ESPRESSO\_TMPDIR. The \texttt{iotk}
toolkit is used to write the file in a XML format, whose definition can
be found in the Developer Manual. In order to use the data directory
on a different machine, you need to convert the binary files to formatted
and back, using the \texttt{bin/iotk} script.

The execution stops if you create a file \texttt{prefix.EXIT} in the working 
directory. NOTA BENE: this is the directory where the program 
is executed, NOT the directory \texttt{outdir} defined in input, where files 
are written. Note that with some versions of MPI, the working directory 
is the directory where the \pwx\ executable is! The advantage of this 
procedure is that all files are properly closed, whereas  just killing 
the process may leave data and output files in unusable state.

\subsection{Electronic structure calculations}
\paragraph{Single-point (fixed-ion) SCF calculation} 
Set \texttt{calculation='scf'} (this is actually the default).
Namelists \&IONS and \&CELL will be ignored. See Example 01.

\paragraph{Band structure calculation}
First perform a SCF calculation as above;
then do a non-SCF calculation with the desired k-point grid and 
number \texttt{nbnd} of bands. 
Use \texttt{calculation='bands'} if you are interested in calculating
only the Kohn-Sham states for the given set of k-points
(e.g. along symmetry lines: see for instance
\texttt{http://www.cryst.ehu.es/cryst/get\_kvec.html}). Specify instead
\texttt{calculation='nscf'} if you are interested in further processing 
of the results of non-SCF calculations (for instance, in DOS calculations).
In the latter case, you should specify a uniform grid of points.
For DOS calculations you should choose \texttt{occupations='tetrahedra'}, 
together with an automatically generated uniform k-point grid 
(card K\_POINTS with option ``automatic'').
Specify \texttt{nosym=.true.} to avoid generation of additional k-points in
low symmetry cases. Variables \texttt{prefix} and \texttt{outdir}, which determine
the names of input or output files, should be the same in the two runs.
See Examples 01, 06, 07,

NOTA BENE:  Since v.4.1, both atomic positions
and the scf
potential are read from the data file so that consistency is guaranteed.

\paragraph{Noncolinear magnetization, spin-orbit interactions}

The following input variables are relevant for noncolinear and
spin-orbit calculations: 
\begin{quote}
      \texttt{noncolin}\\
      \texttt{lspinorb}\\
      \texttt{starting\_magnetization} (one for each type of atoms)
\end{quote}
To make a spin-orbit calculation \texttt{noncolin} must be true. 
If \texttt{starting\_magnetization} is set to zero (or not given) 
the code makes a spin-orbit calculation without spin magnetization 
(it assumes that time reversal symmetry holds and it does not calculate 
the magnetization). The states are still two-component spinors but the
total magnetization is zero. 

If \texttt{starting\_magnetization} is different from zero, it makes a non
collinear spin polarized calculation with spin-orbit interaction. The 
final spin magnetization might be zero or different from zero depending 
on the system. 

Furthermore to make a spin-orbit calculation you must use fully
relativistic pseudopotentials at least for the atoms in which you
think that spin-orbit interaction is large. If all the pseudopotentials 
are scalar
relativistic the calculation becomes equivalent to a noncolinear
calculation without spin orbit. (Andrea Dal Corso, 2007-07-27)
See Example 06 for non-collinear magnetism, Example 07
for spin-orbit interactions.

\paragraph{DFT+U}
DFT+U (formerly known as LDA+U) calculation can be
performed within a simplified rotationally invariant form 
of the $U$ Hubbard correction. See Example 08 
and its README.

\paragraph{Dispersion Interactions (DFT-D)}
For DFT-D (DFT + semiempirical dispersion interactions), see the
description of input variables \texttt{london*}, sample files
\texttt{PW/tests/vdw.*}, and the comments in source file
\texttt{Modules/mm\_dispersion.f90}.

\paragraph{Hartree-Fock and Hybrid functionals}

Since v.5.0, calculations in the Hartree-Fock approximation, or using 
hybrid XC functionals that include some Hartree-Fock exchange, no longer
require a special preprocessing before compilation.
See \texttt{EXX\_example/} and its README file.

\paragraph{Dispersion interaction with non-local functional (vdwDF)}
See example \texttt{vdwDF\_example} 
and references
quoted in file \texttt{README} therein.

\paragraph{Polarization via Berry Phase}
See Example 04, its file README, 
the documentation in the header of 
\texttt{PW/src/bp\_c\_phase.f90}. 

\paragraph{Finite electric fields}
There are two different implementations of macroscopic electric fields
in \pwx: via an external sawtooth potential (input variable
\texttt{tefield=.true.}) and via the modern theory of polarizability
(\texttt{lelfield=.true.}).
The former is useful for surfaces, especially in conjunction
with dipolar corrections (\texttt{dipfield=.true.}):
see \texttt{examples/dipole\_example} for an example of application. 
Electric fields via modern theory of polarization are documented in
example 10. The exact meaning of the related variables, for both
cases, is explained in the general input documentation.

\subsection{Optimization and dynamics}

\paragraph{Structural optimization}
For fixed-cell optimization, specify \texttt{calculation='relax'} and 
add namelist \&IONS. All options for a single SCF calculation apply, 
plus a few others. You
may follow a structural optimization with a non-SCF band-structure
calculation (since v.4.1, you do not need any longer to update the 
atomic positions in the input file for non scf calculation).\\
See Example 02.

\paragraph{Molecular Dynamics} 
Specify \texttt{calculation='md'}, the time step \texttt{dt}, and possibly the number of MD stops \texttt{nstep}.
Use variable \texttt{ion\_dynamics} in namelist \&IONS for a fine-grained control
of the kind of dynamics. Other options for setting the initial
temperature and for thermalization using velocity rescaling are
available. Remember: this is MD on the electronic ground state, not
Car-Parrinello MD.
See Example 03.

\paragraph{Free-energy surface calculations}

Once \PWscf\ is patched with the \texttt{PLUMED} plug-in, it is possible to 
use most PLUMED functionalities by running \PWscf\ as: 
\texttt{./pw.x -plumed} plus the other usual \PWscf\ arguments.
The input file for \texttt{PLUMED} must be found in the specified 
\texttt{outdir} with fixed name \texttt{plumed.dat}.

\paragraph{Variable-cell optimization}

Since v.4.2 the newer BFGS algorithm covers the case of variable-cell
optimization as well. Note however that variable-cell calculations
(both optimization and dynamics) are performed with plane waves and 
G-vectors {\em calculated for the starting cell}. This means that if 
you re-run a self-consistent calculation for the final cell and atomic 
positions using the same cutoff \texttt{ecutwfc} (and/or \texttt{ecutrho} 
if applicable), you may not find exactly the same results, unless your
final and initial cells are very similar, or unless your cutoff(s) are very
high. In order to provide a further check, a last step is performed in
which a scf calculation is performed for the converged structure, with
plane waves and G-vectors {\em calculated for the final cell}. Small 
differences between the two last steps are thus to be expected and give
an estimate of the reliability of the variable-cell optimization.
If you get a large difference, you are likely quite far from convergence
in the plane-wave basis set and you need to increase the cutoff(s).

\paragraph{Variable-cell molecular dynamics}

"A common mistake many new users make is to set the time step \texttt{dt}
improperly to the same order of magnitude as for CP algorithm, or
not setting \texttt{dt} at all. This will produce a ``not evolving dynamics''.
Good values for the original RMW (RM Wentzcovitch) dynamics are 
\texttt{dt} $ = 50 \div 70$. The choice of the cell mass is a delicate matter. An
off-optimal mass will make convergence slower. Too small masses, as
well as too long time steps, can make the algorithm unstable. A good
cell mass will make the oscillation times for internal degrees of
freedom comparable to cell degrees of freedom in non-damped
Variable-Cell MD. Test calculations are advisable before extensive
calculation. I have tested the damping algorithm that I have developed
and it has worked well so far. It allows for a much longer time step
(dt=$100 \div 150$) than the RMW one and is much more stable with very
small cell masses, which is useful when the cell shape, not the
internal degrees of freedom, is far out of equilibrium. It also
converges in a smaller number of steps than RMW." (Info from Cesar Da
Silva: the new damping algorithm is the default since v. 3.1).

\subsection{Direct interface with \texttt{CASINO}} \label{pw2casino_info}

\texttt{PWscf} now supports the Cambridge quantum Monte Carlo program CASINO directly. For more information
on the \texttt{CASINO} code see \texttt{http://www.tcm.phy.cam.ac.uk/\~{}mdt26/casino.html}. 
\texttt{CASINO} may take the output of \texttt{PWSCF} and
'improve it' giving considerably more accurate total energies and other
quantities than DFT is capable of.


\texttt{PWscf} users wishing to learn how to use CASINO may like to attend one
of the annual \texttt{CASINO} summer schools in Mike Towler's "Apuan Alps Centre
for Physics" in Tuscany, Italy. More information can be found at \texttt{http://www.vallico.net/tti/tti.html} 
\paragraph{Practicalities}
The interface between \texttt{PWscf} and \texttt{CASINO} is provided through a file with a
standard format containing geometry, basis set, and orbital coefficients, which
\texttt{PWscf} will produce on demand. For SCF calculations, the name of this file may
be \texttt{pwfn.data}, \texttt{bwfn.data} or \texttt{bwfn.data.b1} depending on user requests (see below).
If the files are produced from an MD run, the files have a suffix \texttt{.0001}, \texttt{.0002},
\texttt{.0003} etc.  corresponding to the sequence of timesteps.

\texttt{CASINO} support is implemented by three routines in the \texttt{PW} directory of the
espresso distribution: 
\begin{itemize}
\item \texttt{pw2casino.f90} : the main routine     
\item \texttt{pw2casino\_write.f90} : writes the \texttt{CASINO} \texttt{xwfn.data} file in various formats
\item \texttt{pw2blip.f90} : does the plane-wave to blip conversion, if requested
\end{itemize}
Relevant behavior of \texttt{PWscf} may be modified through an optional auxiliary input 
file, named \texttt{pw2casino.dat} (see below).

Note that in versions prior to 4.3, this functionality was provided through
separate post-processing utilities available in the PP directory: these are 
no longer supported. For QMC-MD runs, PWSCF etc previously needed to be 
'patched' using the patch script PP/pw2casino-MDloop.sh - this is no longer 
necessary.

\paragraph{How to generate \texttt{xwfn.data} files with \texttt{PWscf}}
Use the '-pw2casino' option when invoking \pwx, e.g.: 
\begin{verbatim}
pw.x -pw2casino < input_file > output_file
\end{verbatim}
The \texttt{xfwn.data} file will then be generated automatically. 

\texttt{PWscf} is capable of doing the plane wave to blip conversion directly (the
'blip' utility provided in the \texttt{CASINO} distribution is not required) and so by
default, \texttt{PWscf} produces the 'binary blip wave function' file \texttt{bwfn.data.b1}

Various options may be modified by providing a file \texttt{pw2casino.dat} in \texttt{outdir}
with the following format: 
\begin{verbatim}
&inputpp
blip_convert=.true.
blip_binary=.true.
blip_single_prec=.false.
blip_multiplicity=1.d0
n_points_for_test=0
/
\end{verbatim}
Some or all of the 5 keywords may be provided, in any order. The default
values are as given above (and these are used if the \texttt{pw2casino.dat} file is
not present.

The meanings of the keywords are as follows: 
\begin{description}
\item [blip\_convert]: reexpand the converged plane-wave orbitals in localized blip 
functions prior to writing the \texttt{CASINO} wave function file. This is almost
always done, since wave functions expanded in blips are considerably more
efficient in quantum Monte Carlo calculations. If \texttt{blip\_convert=.false.}
a pwfn.data file is produced (orbitals expanded in plane waves); if 
\texttt{blip\_convert=.true.}, either a \texttt{bwfn.data file} or a \texttt{bwfn.data.b1} file is 
produced, depending on the value of \texttt{blip\_binary} (see below).

\item [blip\_binary]: if true, and if \texttt{blip\_convert} is also true, write the blip wave function
as an unformatted binary \texttt{bwfn.data.b1} file. This is much smaller than
the formatted \texttt{bwfn.data} file, but is not generally portable across
all machines.

\item [blip\_single\_prec]: if \texttt{.false.} the orbital coefficients in \texttt{bwfn.data(.b1)} are written out in double
precision; if the user runs into hardware limits \texttt{blip\_single\_prec} can be
set to \texttt{.true.} in which case the coefficients are written in single
precision, reducing the memory and disk requirements at the cost of a small
amount of accuracy..

\item [blip\_multiplicity]: the quality of the blip expansion (i.e., the fineness of the blip grid) can be
improved by increasing the grid multiplicity parameter given by this keyword.
Increasing the grid multiplicity results in a greater number of blip
coefficients and therefore larger memory requirements and file size, but the
CPU time should be unchanged. For very accurate work, one may want to 
experiment with grid multiplicity larger that 1.0. Note, however, that it 
might be more efficient to keep the grid multiplicity to 1.0 and increase the 
plane wave cutoff instead.

\item [n\_points\_for\_test]: if this is set to a positive integer greater than zero, \texttt{PWscf} will sample 
the wave function, the Laplacian and the gradient at a large number of
random points in the simulation cell and compute the overlap of the
blip orbitals with the original plane-wave orbitals:
$$
\alpha = {<BW|PW> \over  \sqrt{<BW|BW><PW|PW>}}
$$
The closer $\alpha$ is to 1, the better the blip representation.  By increasing
\texttt{blip\_multiplicity}, or by increasing the plane-wave cutoff, one ought to be 
able to make $\alpha$ as close to 1 as desired. The number of random points used 
is given by \texttt{n\_points\_for\_test}.
\end{description}

Finally, note that DFT trial wave functions produced by \texttt{PWSCF}
must be generated using the same pseudopotential as in the subsequent QMC 
calculation. This requires the use of tools to switch between the different 
file formats used by the two codes.

\texttt{CASINO} uses the `\texttt{CASINO} tabulated format', \texttt{PWSCF} officially supports 
the UPFv2 format (though it will read other `deprecated' formats).
This can be done through the `casino2upf' and `upf2casino' tools included in the upftools directory (see the upftools/README file for instructions). An alternative converter `casinogon' is included in the \texttt{CASINO} distribution which produces the deprecated GON format but which can be useful when using non-standard grids.

\section{Performances}

\subsection{Execution time}

The following is a rough estimate of the complexity of a plain
scf calculation with \pwx, for NCPP. USPP and PAW 
give raise additional terms to be calculated, that may add from a 
few percent 
up to 30-40\% to execution time. For phonon calculations, each of the
$3N_{at}$ modes requires a time of the same order of magnitude of
self-consistent calculation in the same system (possibly times a small multiple). 
For \cpx, each time step takes something in the order of
$T_h + T_{orth} + T_{sub}$ defined below.

The time required for the self-consistent solution at fixed ionic
positions, $T_{scf}$ , is:
$$T_{scf} = N_{iter} T_{iter} + T_{init}$$
where $N_{iter}$  = number of self-consistency iterations (\texttt{niter}), 
$T_{iter}$ =
time for a single iteration, $T_{init}$ = initialization time
(usually much smaller than the first term).

The time required for a single self-consistency iteration $T_{iter}$ is:
$$T_{iter} = N_k T_{diag} +T_{rho} + T_{scf}$$
where $N_k$ = number of k-points, $T_{diag}$ = time per 
Hamiltonian iterative diagonalization, $T_{rho}$ = time for charge density 
calculation, $T_{scf}$ = time for Hartree and XC potential
calculation.
    
The time for a Hamiltonian iterative diagonalization $T_{diag}$ is:
$$T_{diag} = N_h T_h + T_{orth} + T_{sub}$$
where $N_h$ = number of $H\psi$ products needed by iterative diagonalization,
$T_h$ = time per $H\psi$ product, $T_{orth}$ = CPU time for 
orthonormalization, $T_{sub}$ = CPU time for subspace diagonalization.
    
The time $T_h$ required for a $H\psi$ product is
$$T_h = a_1 M N + a_2 M N_1 N_2 N_3 log(N_1 N_2 N_3 ) + a_3 M P N. $$
The first term comes from the kinetic term and is usually much smaller
than the others. The second and third terms come respectively from local
and nonlocal potential. $a_1, a_2, a_3$ are prefactors (i.e.
small numbers ${\cal O}(1)$), $M$ = number of valence
bands (\texttt{nbnd}), $N$ = number of PW (basis set dimension: \texttt{npw}), $N_1, N_2, N_3$ =
dimensions of the FFT grid for wavefunctions (\texttt{nr1s}, \texttt{nr2s},
\texttt{nr3s}; $N_1 N_2 N_3 \sim 8N$ ), 
P = number of pseudopotential projectors, summed on all atoms, on all values of the
angular momentum $l$, and $m = 1, . . . , 2l + 1$.

The time $T_{orth}$ required by orthonormalization is
$$T_{orth} = b_1 N M_x^2$$ 
and the time $T_{sub}$ required by subspace diagonalization is
$$T_{sub} = b_2 M_x^3$$
where $b_1$ and $b_2$ are prefactors, $M_x$ = number of trial wavefunctions 
(this will vary between $M$ and $2\div4 M$, depending on the algorithm).
    
The time $T_{rho}$ for the calculation of charge density from wavefunctions is
$$T_{rho} = c_1 M N_{r1} N_{r2}N_{r3} log(N_{r1} N_{r2} N_{r3}) + 
            c_2 M N_{r1} N_{r2} N_{r3} + T_{us}$$
where $c_1, c_2, c_3$ are prefactors, $N_{r1}, N_{r2}, N_{r3}$ =
dimensions of the FFT grid for charge density (\texttt{nr1},
\texttt{nr2}, \texttt{nr3}; $N_{r1} N_{r2} N_{r3} \sim 8N_g$,
where $N_g$ = number of G-vectors for the charge density,
\texttt{ngm}), and 
$T_{us}$ = time required by PAW/USPPs contribution (if any).
Note that for NCPPs the FFT grids for charge and
wavefunctions are the same.
 
The time $T_{scf}$ for calculation of potential from charge density is
$$T_{scf} = d_2 N_{r1} N_{r2} N_{r3} + d_3 N_{r1} N_{r2} N_{r3} 
            log(N_{r1} N_{r2} N_{r3} )$$
where $d_1, d_2$ are prefactors.

The above estimates are for serial execution. In parallel execution,
each contribution may scale in a different manner with the number of processors (see below).

\subsection{Memory requirements}

A typical self-consistency or molecular-dynamics run requires a maximum
memory in the order of $O$ double precision complex numbers, where
$$ O = m M N + P N + p N_1 N_2 N_3 + q N_{r1} N_{r2} N_{r3}$$
with $m, p, q$ = small factors; all other variables have the same meaning as
above. Note that if the $\Gamma-$point only ($k=0$) is used to sample the 
Brillouin Zone, the value of N will be cut into half.

The memory required by the phonon code follows the same patterns, with
somewhat larger factors $m, p, q$.

\subsection{File space requirements}

A typical \pwx\ run will require an amount of temporary disk space in the
order of O double precision complex numbers:
$$O = N_k M N + q N_{r1} N_{r2}N_{r3}$$
where $q = 2\times$ \texttt{mixing\_ndim} (number of iterations used in 
self-consistency, default value = 8) if \texttt{disk\_io} is set to 'high'; q = 0 
otherwise.

\subsection{Parallelization issues}
\label{SubSec:badpara}

\pwx\ can run in principle on any number of processors.
The effectiveness of parallelization is ultimately judged by the 
''scaling'', i.e. how the time needed to perform a job scales
 with the number of processors, and depends upon:
\begin{itemize}
\item the size and type of the system under study;
\item the judicious choice of the various levels of parallelization 
(detailed in Sec.\ref{SubSec:para});
\item the availability of fast interprocess communications (or lack of it).
\end{itemize}
Ideally one would like to have linear scaling, i.e. $T \sim T_0/N_p$ for 
$N_p$ processors, where $T_0$ is the estimated time for serial execution.
 In addition, one would like to have linear scaling of
the RAM per processor: $O_N \sim O_0/N_p$, so that large-memory systems
fit into the RAM of each processor.

Parallelization on k-points:
\begin{itemize}
\item guarantees (almost) linear scaling if the number of k-points
is a multiple of the number of pools;
\item requires little communications (suitable for ethernet communications);
\item does not reduce the required memory per processor (unsuitable for 
large-memory jobs).
\end{itemize}
Parallelization on PWs:
\begin{itemize}
\item yields good to very good scaling, especially if the number of processors
in a pool is a divisor of $N_3$ and $N_{r3}$ (the dimensions along the z-axis 
of the FFT grids, \texttt{nr3} and \texttt{nr3s}, which coincide for NCPPs);
\item requires heavy communications (suitable for Gigabit ethernet up to 
4, 8 CPUs at most, specialized communication hardware needed for 8 or more
processors );
\item yields almost linear reduction of memory per processor with the number
of processors in the pool.
\end{itemize}

A note on scaling: optimal serial performances are achieved when the data are
as much as possible kept into the cache. As a side effect, PW
parallelization may yield superlinear (better than linear) scaling,
thanks to the increase in serial speed coming from the reduction of data size 
(making it easier for the machine to keep data in the cache).

VERY IMPORTANT: For each system there is an optimal range of number of processors on which to 
run the job.  A too large number of processors will yield performance 
degradation. If the size of pools is especially delicate: $N_p$ should not 
exceed $N_3$ and $N_{r3}$, and should ideally be no larger than
$1/2\div1/4 N_3$ and/or $N_{r3}$. In order to increase scalability,
it is often convenient to 
further subdivide a pool of processors into ''task groups''.
When the number of processors exceeds the number of FFT planes, 
data can be redistributed to "task groups" so that each group 
can process several wavefunctions at the same time.

The optimal number of processors for "linear-algebra"
parallelization, taking care of multiplication and diagonalization 
of $M\times M$ matrices, should be determined by observing the
performances of \texttt{cdiagh/rdiagh} (\pwx) or \texttt{ortho} (\cpx)
for different numbers of processors in the linear-algebra group
(must be a square integer).

Actual parallel performances will also depend on the available software 
(MPI libraries) and on the available communication hardware. For
PC clusters, OpenMPI (\texttt{http://www.openmpi.org/}) seems to yield better 
performances than other implementations (info by Kostantin Kudin). 
Note however that you need a decent communication hardware (at least 
Gigabit ethernet) in order to have acceptable performances with 
PW parallelization. Do not expect good scaling with cheap hardware: 
PW calculations are by no means an "embarrassing parallel" problem.
   
Also note that multiprocessor motherboards for Intel Pentium CPUs typically 
have just one memory bus for all processors. This dramatically
slows down any code doing massive access to memory (as most codes 
in the \qe\ distribution do) that runs on processors of the same
motherboard.

\subsection{Understanding the time report}

The time report printed at the end of a \pwx\ run contains a lot of useful 
information that can be used to understand bottlenecks and improve 
performances.

\subsubsection{Serial execution}
The following applies to calculations taking a sizable amount of time
(at least minutes): for short calculations (seconds), the time spent in 
the various initializations dominates. Any discrepancy with the following
picture signals some anomaly.

\begin{itemize}
\item
  For a typical job with norm-conserving PPs, the total (wall) time is mostly 
  spent in routine "electrons", calculating the self-consistent solution. 
\item
  Most of the time spent in "electrons" is used by routine "c\_bands", 
  calculating Kohn-Sham states. "sum\_band" (calculating the charge density),
  "v\_of\_rho" (calculating the potential), "mix\_rho" (charge density mixing)
  should take a small fraction of the time.
\item
  Most of the time spent in "c\_bands" is used by routines "cegterg" (k-points)
  or "regterg" (Gamma-point only), perfoming iterative diagonalization of
  the Kohn-Sham Hamiltonian in the PW basis set. 
\item
  Most of the time spent in "*egterg" is used by routine "h\_psi",
  calculating $H\psi$ products. "cdiaghg" (k-points) or "rdiaghg" (Gamma-only), 
  performing subspace diagonalization, should take only a small fraction.
\item
  Among the "general routines", most of the time is spent in FFT on Kohn-Sham
  states: "fftw", and to a smaller extent in other FFTs, "fft" and "ffts", 
  and in "calbec", calculating $\langle\psi|\beta\rangle$ products. 
\item
  Forces and stresses typically take a fraction of the order of 10 to 20\%
  of the total time.
\end{itemize}
For PAW and Ultrasoft PP, you will see a larger contribution by "sum\_band" 
and a nonnegligible "newd" contribution to the time spent in "electrons", 
but the overall picture is unchanged. You may drastically reduce the
overhead of Ultrasoft PPs by using input option "tqr=.true.".

\subsubsection{Parallel execution}

The various parallelization levels should be used wisely in order to 
achieve good results. Let us summarize the effects of them on CPU:
\begin{itemize}
\item
  Parallelization on FFT speeds up (with varying efficiency) almost 
  all routines, with the notable exception of "cdiaghg" and "rdiaghg".
\item
  Parallelization on k-points speeds up (almost linearly) "c\_bands" and 
  called routines; speeds up partially "sum\_band"; does not speed up
  at all "v\_of\_rho", "newd", "mix\_rho".
\item
  Linear-algebra parallelization speeds up (not always) "cdiaghg" and "rdiaghg" 
\item
  "task-group" parallelization speeds up "fftw"
\item
  OpenMP parallelization speeds up "fftw", plus selected parts of the 
  calculation, plus (depending on the availability of OpenMP-aware
  libraries) some linear algebra operations
\end{itemize}
and on RAM:
\begin{itemize}
\item
  Parallelization on FFT distributes most arrays across processors
  (i.e. all G-space and R-spaces arrays) but not all of them (in
  particular, not subspace Hamiltonian and overlap matrices)
\item
  Linear-algebra parallelization also distributes subspace Hamiltonian
  and overlap matrices.
\item
  All other parallelization levels do not distribute any memory
\end{itemize}
In an ideally parallelized run, you should observe the following:
\begin{itemize}
\item
  CPU and wall time do not differ by much
\item
  Time usage is still dominated by the same routines as for the serial run
\item
  Routine "fft\_scatter" (called by parallel FFT) takes a sizable part of
  the time spent in FFTs but does not dominate it.
\end{itemize}

\paragraph{ Quick estimate of parallelization parameters}

You need to know
\begin{itemize}
\item 
  the number of k-points, $N_k$
\item
  the third dimension of the (smooth) FFT grid, $N_3$
\item
  the number of Kohn-Sham states, $M$
\end{itemize}
These data allow to set bounds on parallelization:
\begin{itemize}
\item 
  k-point parallelization is limited to $N_k$ processor pools: 
  \texttt{-npool Nk}
\item
  FFT parallelization shouldn't exceed $N_3$ processors, i.e. if you
  run with \texttt{-npool Nk}, use $N=N_k\times N_3$ MPI processes at most (\texttt{mpirun -np N ...})
\item
  Unless $M$ is a few hundreds or more, don't bother using linear-algebra
  parallelization
\end{itemize}
You will need to experiment a bit to find the best compromise. In order
to have good load balancing among MPI processes, the number of k-point
pools should be an integer divisor of $N_k$; the number of processors for
FFT parallelization should be an integer divisor of $N_3$. 

\paragraph{Typical symptoms of bad/inadequate parallelization}

\begin{itemize}
\item 
{\em  a large fraction of time is spent in "v\_of\_rho", "newd", "mix\_rho"}, or\\
{\em  the time doesn't scale well or doesn't scale at all by increasing the 
  number of processors for k-point parallelization.}  Solution:
\begin{itemize}
\item 
  use (also) FFT parallelization if possible
\end{itemize}
\item
{\em  a disproportionate time is spent in "cdiaghg"/"rdiaghg".} Solutions:
\begin{itemize}
\item 
   use (also) k-point parallelization if possible
\item 
   use linear-algebra parallelization, with scalapack if possible.
\end{itemize}
\item
{\em a disproportionate time is spent in "fft\_scatter"}, or
{\em in "fft\_scatter" the difference between CPU and wall time is large.} Solutions:
\begin{itemize}
\item 
    if you do not have fast (better than Gigabit ethernet) communication
    hardware, do not try FFT parallelization on more than 4 or 8 procs.
\item 
    use (also) k-point parallelization if possible
\end{itemize}
\item
{\em  the time doesn't scale well or doesn't scale at all by increasing the 
  number of processors for FFT parallelization.}
    Solutions:
\begin{itemize}
\item 
    use "task groups": try command-line option \texttt{-ntg 4} or
    \texttt{-ntg 8}. This may improve your scaling.
\end{itemize}
\end{itemize}

\section{Troubleshooting}

\paragraph{pw.x says 'error while loading shared libraries' or
  'cannot open shared object file' and does not start} 
Possible reasons:
\begin{itemize}
\item If you are running on the same machines on which the code was
  compiled, this is a library configuration problem. The solution is
  machine-dependent. On Linux, find the path to the missing libraries;
  then either add it to file \texttt{/etc/ld.so.conf} and run \texttt{ldconfig}
   (must be
  done as root), or add it to variable LD\_LIBRARY\_PATH and export
  it. Another possibility is to load non-shared version of libraries
  (ending with .a)  instead of shared ones (ending with .so). 
\item If you are {\em not} running on the same machines on which the
  code was compiled: you need either to have the same shared libraries
  installed on both machines, or to load statically all libraries
  (using appropriate \configure\ or loader options). The same applies to
  Beowulf-style parallel machines: the needed shared libraries must be
  present on all PCs. 
\end{itemize}

\paragraph{errors in examples with parallel execution}

If you get error messages in the example scripts -- i.e. not errors in
the codes -- on a parallel machine, such as e.g.: 
{\em run example: -n: command not found}
you may have forgotten 
the " " in the definitions of PARA\_PREFIX and PARA\_POSTFIX.

\paragraph{pw.x prints the first few lines and then nothing happens
  (parallel execution)} 
If the code looks like it is not reading from input, maybe
it isn't: the MPI libraries need to be properly configured to accept input
redirection. Use \texttt{pw.x -inp} and the input file name (see Sec.\ref{SubSec:para}), or inquire with
your local computer wizard (if any). Since v.4.2, this is for sure the
reason if the code stops at {\em Waiting for input...}.

\paragraph{pw.x stops with error while reading data}
There is an error in the input data, typically a misspelled namelist 
variable, or an empty input file.
Unfortunately with most compilers the code just reports {\em Error while
reading XXX namelist} and no further useful information.
Here are some more subtle sources of trouble:
\begin{itemize}
\item Out-of-bound indices in dimensioned variables read in the namelists;
\item Input data files containing \^{}M (Control-M) characters at the end
  of lines, or non-ASCII characters (e.g. non-ASCII quotation marks,
  that at a first glance may look the same as the ASCII
  character). Typically, this happens with files coming from Windows
  or produced with "smart" editors.  
\end{itemize}
Both may cause the code to crash with rather mysterious error messages.
If none of the above applies and the code stops at the first namelist
(\&CONTROL) and you are running in parallel, see the previous item.

\paragraph{pw.x mumbles something like {\em cannot recover} or 
{\em error reading recover file}} 
You are trying to restart from a previous job that either
produced corrupted files, or did not do what you think it did. No luck: you
have to restart from scratch.

\paragraph{pw.x stops with {\em inconsistent DFT} error}
As a rule, the flavor of DFT used in the calculation should be the
same as the one used in the generation of pseudopotentials, which
should all be generated using the same flavor of DFT. This is actually enforced: the
type of DFT is read from pseudopotential files and it is checked that the same DFT
is read from all PPs. If this does not hold, the code stops with the
above error message. Use -- at your own risk -- input variable 
\texttt{input\_dft} to force the usage of the DFT you like.

\paragraph{pw.x stops with error in cdiaghg or rdiaghg}
Possible reasons for such behavior are not always clear, but they
typically fall into one of the following cases:
\begin{itemize}
\item serious error in data, such as bad atomic positions or bad
  crystal structure/supercell; 
\item a bad pseudopotential, typically with a ghost, or a USPP giving
  non-positive charge density, leading to a violation of positiveness
  of the S matrix appearing in the USPP formalism;  
\item a failure of the algorithm performing subspace
  diagonalization. The LAPACK algorithms used by \texttt{cdiaghg}
  (for generic k-points) or \texttt{rdiaghg} (for $\Gamma-$only case)
  are
  very robust and extensively tested. Still, it may seldom happen that
  such algorithms fail. Try to use conjugate-gradient diagonalization
  (\texttt{diagonalization='cg'}), a slower but very robust algorithm, and see
  what happens. 
\item buggy libraries. Machine-optimized mathematical libraries are
  very fast but sometimes not so robust from a numerical point of
  view.  Suspicious behavior: you get an error that is not
  reproducible on other architectures or that disappears if the
  calculation is repeated with even minimal changes in
  parameters. Known cases: HP-Compaq alphas with cxml libraries, Mac
  OS-X with system BLAS/LAPACK. Try to use compiled BLAS and LAPACK
  (or better, ATLAS) instead of machine-optimized libraries. 
\end{itemize}

\paragraph{pw.x crashes with no error message at all}
This happens quite often in parallel execution, or under a batch
queue, or if you are writing the output to a file. When the program
crashes, part of the output, including the error message, may be lost,
or hidden into error files where nobody looks into. It is the fault of
the operating system, not of the code. Try to run interactively 
and to write to the screen. If this doesn't help, move to next point.

\paragraph{pw.x crashes with {\em segmentation fault} or similarly
  obscure messages} 
Possible reasons:
\begin{itemize}
\item  too much RAM memory or stack  requested (see next item). 
\item if you are using highly optimized mathematical libraries, verify
  that they are designed for your hardware.
\item If you are using aggressive optimization in compilation, verify
that you are using the appropriate options for your machine
\item The executable was not properly compiled, or was compiled on
a different and incompatible environment.
\item buggy compiler or libraries: this is the default explanation if you
have problems with the provided tests and examples.
\end{itemize}

\paragraph{pw.x works for simple systems, but not for large systems
  or whenever more RAM is needed}  
Possible solutions:
\begin{itemize}
\item increase the amount of RAM you are authorized to use (which may
  be much smaller than the available RAM). Ask your system
  administrator if you don't know what to do. In some cases the 
  stack size can be a source of problems: if so, increase it with command 
  \texttt{limits} or \texttt{ulimit}).
\item reduce \texttt{nbnd} to the strict minimum, or reduce the cutoffs, or the
  cell size , or a combination of them
\item  use conjugate-gradient (\texttt{diagonalization='cg'}: slow but very
  robust): it requires less memory than the default Davidson
  algorithm.  If you stick to the latter, use \texttt{diago\_david\_ndim=2}.
\item in parallel execution, use more processors, or use the same
  number of processors with less pools. Remember that parallelization
  with respect to k-points (pools) does not distribute memory:
  parallelization with respect to R- (and G-) space does. 
\item buggy or weird-behaving compiler.
\end{itemize}

\paragraph{pw.x crashes with {\em error in davcio}}
\texttt{davcio} is the routine that performs most of the I/O operations (read
from disk and write to disk) in \pwx; {\em error in davcio} means a
failure of an I/O operation. 
\begin{itemize}
\item If the error is reproducible and happens at the beginning of a
  calculation: check if you have read/write permission to the scratch
  directory specified in variable \texttt{outdir}. Also: check if there is
  enough free space available on the disk you are writing to, and
  check your disk quota (if any).
\item If the error is irreproducible: your might have flaky disks; if
  you are writing via the network using NFS (which you shouldn't do
  anyway), your network connection might be not so stable, or your 
  NFS implementation is unable to work under heavy load 
\item If it happens while restarting from a previous calculation: you
  might be restarting from the wrong place, or from wrong  data, or
  the files might be corrupted. 
\item If you are running two or more instances of \pwx\ at
  the same time, check if you are using the same file names in the 
  same temporary directory. For instance, if you submit a series of
  jobs to a batch queue, do not use the same \texttt{outdir} and
  the same \texttt{prefix}, unless you are sure that one job doesn't
  start before a preceding one has finished.
\end{itemize}

\paragraph{pw.x crashes in parallel execution with an obscure message
  related to MPI errors} 
Random crashes due to MPI errors have often been reported, typically
in Linux PC clusters. We cannot rule out the possibility that bugs in
\qe\ cause such behavior, but we are quite confident that
the most likely explanation is a hardware problem (defective RAM  
for instance) or a software bug (in MPI libraries, compiler, operating
system). 

Debugging a parallel code may be difficult, but you should at least
verify if your problem is reproducible on different
architectures/software configurations/input data sets, and if  
there is some particular condition that activates the bug. If this
doesn't seem to happen, the odds are that the problem is not in
\qe. You may still report your problem, 
but consider that reports like {\em it crashes with...(obscure MPI error)}
contain 0 bits of information and are likely to get 0 bits of answers.

\paragraph{pw.x stops with error message {\em the system is metallic,
 specify occupations}} 
You did not specify state occupations, but you need to, since your
system appears to have an odd number of electrons. The variable
controlling how metallicity is treated is \texttt{occupations} in namelist
\&SYSTEM. The default, \texttt{occupations='fixed'}, occupies the lowest
(N electrons)/2 states and works only for insulators with a gap. In all other
cases, use \texttt{'smearing'} (\texttt{'tetrahedra'} for DOS calculations). 
See input reference documentation for more details.

\paragraph{pw.x stops with {\em internal error: cannot bracket Ef}}
Possible reasons:
\begin{itemize}
\item serious error in data, such as bad number of electrons,
  insufficient number of bands, absurd value of broadening; 
\item the Fermi energy is found by bisection assuming that the
  integrated DOS N(E ) is an increasing function of the energy. This
  is not guaranteed for Methfessel-Paxton smearing of order 1 and can
  give problems when very few k-points are used. Use some other
  smearing function: simple Gaussian broadening or, better,
  Marzari-Vanderbilt 'cold smearing'. 
\end{itemize}

\paragraph{pw.x yields {\em internal error: cannot bracket Ef} message 
but does not stop} 
This may happen under special circumstances when you are calculating
the band structure for selected high-symmetry lines. The message
signals that occupations and Fermi energy are not correct (but
eigenvalues and eigenvectors are). Remove \texttt{occupations='tetrahedra'}
in the input data to get rid of the message. 

\paragraph{pw.x runs but nothing happens}
Possible reasons:
\begin{itemize}
\item in parallel execution, the code died on just one
  processor. Unpredictable behavior may follow. 
\item in serial execution, the code encountered a floating-point error
  and goes on producing NaNs (Not a Number) forever unless exception
  handling is on (and usually it isn't). In both cases, look for one
  of the reasons given above. 
\item maybe your calculation will take more time than you expect.
\end{itemize}

\paragraph{pw.x yields weird results}
If results are really weird (as opposed to misinterpreted):
\begin{itemize}
\item if this happens after a change in the code or in compilation or
  preprocessing options, try \texttt{make clean}, recompile. The \texttt{make}
  command should take care of all dependencies, but do not rely too
  heavily on it. If the problem persists, recompile with
  reduced optimization level.  
\item maybe your input data are weird.
\end{itemize}

\paragraph{FFT grid is machine-dependent}
Yes, they are! The code automatically chooses the smallest grid that
is compatible with the 
specified cutoff in the specified cell, and is an allowed value for the FFT
library used. Most FFT libraries are implemented, or perform well, only
with dimensions that factors into products of small numbers (2, 3, 5 typically,
sometimes 7 and 11). Different FFT libraries follow different rules and thus
different dimensions can result for the same system on different machines (or
even on the same machine, with a different FFT). See function allowed in
\texttt{Modules/fft\_scalar.f90}.

As a consequence, the energy may be slightly different on different machines. 
The only piece that explicitly depends on the grid parameters is
the XC part of the energy that is computed numerically on the grid. The
differences should be small, though, especially for LDA calculations.

Manually setting the FFT grids to a desired value is possible, but slightly
tricky, using input variables \texttt{nr1}, \texttt{nr2}, \texttt{nr3} and 
\texttt{nr1s}, \texttt{nr2s}, \texttt{nr3s}. The
code will still increase them if not acceptable. Automatic FFT grid 
dimensions are slightly overestimated, so one may try {\em very carefully}
to reduce
them a little bit. The code will stop if too small values are required, it will
waste CPU time and memory for too large values.
    
Note that in parallel execution, it is very convenient to have FFT grid
dimensions along $z$ that are a multiple of the number of processors.

\paragraph{pw.x does not find all the symmetries you expected} 
\pwx\ determines first the symmetry operations (rotations) of the
Bravais lattice; then checks which of these are symmetry operations of
the system (including if needed fractional translations). This is done
by rotating (and translating if needed) the atoms in the unit cell and
verifying if the rotated unit cell coincides with the original one.

Assuming that your coordinates are correct (please carefully check!),
you may not find all the symmetries you expect because:
\begin{itemize}
\item the number of significant figures in the atomic positions is not
  large enough. In file \texttt{PW/eqvect.f90}, the variable \texttt{accep} is used to
  decide whether a rotation is a symmetry operation. Its current value
  ($10^{-5}$) is quite strict: a rotated atom must coincide with
  another atom to 5 significant digits. You may change the value of
  accep and recompile. 
\item they are not acceptable symmetry operations of the Bravais
  lattice. This is the case for C$_{60}$, for instance: the $I_h$
  icosahedral group of C$_{60}$ contains 5-fold rotations that are
  incompatible with translation symmetry.  
\item  the system is rotated with respect to symmetry axis. For
  instance: a C$_{60}$ molecule in the fcc lattice will have 24
  symmetry operations ($T_h$ group) only if the double bond is
  aligned along one of the crystal axis; if C$_{60}$ is rotated
  in some arbitrary way, \pwx\ may not find any symmetry, apart from
  inversion. 
\item they contain a fractional translation that is incompatible with
  the FFT grid (see next paragraph). Note that if you change cutoff or
  unit cell volume, the automatically computed FFT grid changes, and
  this may explain changes in symmetry (and in the number of k-points
  as a consequence) for no apparent good reason (only if you have
  fractional translations in the system, though). 
\item a fractional translation, without rotation, is a symmetry
  operation of the system. This means that the cell is actually a
  supercell. In this case, all symmetry operations containing
  fractional translations are disabled. The reason is that in this
  rather exotic case there is no simple way to select those symmetry
  operations forming a true group, in the mathematical sense of the
  term. 
\end{itemize}

\paragraph{{\em Warning: symmetry operation \# N not allowed}}
This is not an error.  If a symmetry operation contains a fractional
translation that is incompatible with the FFT grid, it is discarded in
order to prevent problems with symmetrization. Typical fractional 
translations are 1/2 or 1/3 of a lattice vector. If the FFT grid
dimension along that direction is not divisible respectively by 2 or
by 3, the symmetry operation will not transform the FFT grid into
itself. 

\paragraph{Self-consistency is slow or does not converge at all}

Bad input data will often result in bad scf convergence. Please 
carefully check your structure first, e.g. using XCrySDen.
 
Assuming that your input data is sensible :
\begin{enumerate}
\item Verify if your system is metallic or is close to a metallic
  state, especially if you have few k-points. If the highest occupied
  and lowest unoccupied state(s) keep exchanging place during
  self-consistency, forget about reaching convergence. A typical sign
  of such behavior is that the self-consistency error goes down, down,
  down, than all of a sudden up again, and so on. Usually one can
  solve the problem by adding a few empty bands and a small
  broadening. 
\item Reduce \texttt{mixing\_beta} to $\sim 0.3\div
  0.1$ or smaller. Try the \texttt{mixing\_mode} value that is more
  appropriate for your problem. For slab geometries used in surface
  problems or for elongated cells,  \texttt{mixing\_mode='local-TF'}
  should be the better choice, dampening "charge sloshing". You may
  also try to increase \texttt{mixing\_ndim} to more than 8 (default
  value). Beware: this will increase the amount of memory you need. 
\item Specific to USPP: the presence of negative charge density
  regions due to either the pseudization procedure of the augmentation
  part or to truncation at finite cutoff may give convergence
  problems. Raising the \texttt{ecutrho} cutoff for charge density will
  usually help.
\end{enumerate}

\paragraph{I do not get the same results in different machines!}
If the difference is small, do not panic. It is quite normal for
iterative methods to reach convergence through different paths as soon
as anything changes. In particular, between serial and parallel
execution there are operations that are not performed in the same
order. As the numerical accuracy of computer numbers is finite, this
can yield slightly different results. 

It is also normal that the total energy converges to a better accuracy
than its terms, since only the sum is variational, i.e. has a minimum
in correspondence to ground-state charge density. Thus if the
convergence threshold is for instance $10^{-8}$, you get 8-digit
accuracy on the total energy, but one or two less on other terms
(e.g. XC and Hartree energy). It this is a problem for you, reduce the
convergence threshold for instance to  $10^{-10}$ or  $10^{-12}$. The
differences should go away (but it will probably take a few more
iterations to converge). 

\paragraph{Execution time is time-dependent!}
Yes it is! On most machines and on
most operating systems, depending on machine load, on communication load
(for parallel machines), on various other factors (including maybe the phase
of the moon), reported execution times may vary quite a lot for the same job.

\paragraph{{\em Warning : N eigenvectors not converged}}
This is a warning message that can be safely ignored if it is not
present in the last steps of self-consistency. If it is still present
in the last steps of self-consistency, and if the number of
unconverged eigenvector is a significant part of the total, it may
signal serious trouble in self-consistency (see next point) or
something badly wrong in input data.

\paragraph{{\em Warning : negative or imaginary charge...}, or 
{\em ...core  charge ...}, or {\em npt with rhoup$<0$...} or {\em rho dw$<0$...}} 
These are warning messages that can be safely ignored unless the
negative or imaginary charge is sizable, let us say of the order of
0.1. If it is, something seriously wrong is going on. Otherwise, the
origin of the negative charge is the following. When one transforms a
positive function in real space to Fourier space and truncates at some
finite cutoff, the positive function is no longer guaranteed to be
positive when transformed back to real space. This happens only with
core corrections and with USPPs. In some cases it
may be a source of trouble (see next point) but it is usually solved
by increasing the cutoff for the charge density.

\paragraph{Structural optimization is slow or does not converge or ends 
with a mysterious bfgs error}
Typical structural optimizations, based on the BFGS algorithm,
converge to the default thresholds ( etot\_conv\_thr and
forc\_conv\_thr ) in 15-25 BFGS steps (depending on the 
starting configuration). This may not happen when your
system is characterized by "floppy" low-energy modes, that make very
difficult (and of little use anyway) to reach a well converged structure, no
matter what. Other possible reasons for a problematic convergence are listed
below.
    
Close to convergence the self-consistency error in forces may become large
with respect to the value of forces. The resulting mismatch between forces
and energies may confuse the line minimization algorithm, which assumes
consistency between the two. The code reduces the starting self-consistency
threshold conv thr when approaching the minimum energy configuration, up
to a factor defined by \texttt{upscale}. Reducing \texttt{conv\_thr}
(or increasing \texttt{upscale})
yields a smoother structural optimization, but if \texttt{conv\_thr} becomes too small,
electronic self-consistency may not converge. You may also increase variables
\texttt{etot\_conv\_thr} and \texttt{forc\_conv\_thr} that determine the threshold for
convergence (the default values are quite strict).
    
A limitation to the accuracy of forces comes from the absence of perfect
translational invariance. If we had only the Hartree potential, our PW
calculation would be translationally invariant to machine
precision. The presence of an XC potential
introduces Fourier components in the potential that are not in our
basis set. This loss of precision (more serious for gradient-corrected
functionals) translates into a slight but detectable loss 
of translational invariance (the energy changes if all atoms are displaced by
the same quantity, not commensurate with the FFT grid). This sets a limit
to the accuracy of forces. The situation improves somewhat by increasing
the \texttt{ecutrho} cutoff.

\paragraph{pw.x stops during variable-cell optimization in
  checkallsym with {\em non orthogonal operation} error} 
Variable-cell optimization may occasionally break the starting
symmetry of the cell. When this happens, the run is stopped because
the number of k-points calculated for the starting configuration may
no longer be suitable. Possible solutions: 
\begin{itemize}
\item start with a nonsymmetric cell;
\item use a symmetry-conserving algorithm: the Wentzcovitch algorithm
  (\texttt{cell dynamics='damp-w'}) should not break the symmetry. 
\end{itemize}

\subsection{Compilation problems with \texttt{PLUMED}}

\paragraph{xlc compiler}
If you get an error message like:
\begin{verbatim}
Operation between types "char**" and "int" is not allowed.
\end{verbatim}
change in file \texttt{clib/metadyn.h}
\begin{verbatim}
#define snew(ptr,nelem) (ptr)= (nelem==0 ? NULL : (typeof(ptr)) calloc(nelem, sizeof(*(ptr))))
#define srenew(ptr,nelem) (ptr)= (typeof(ptr)) realloc(ptr,(nelem)*sizeof(*(ptr)))
\end{verbatim}
with
\begin{verbatim}
#define snew(ptr,nelem) (ptr)= (nelem==0 ? NULL : (void*) calloc(nelem, sizeof(*(ptr))))
#define srenew(ptr,nelem) (ptr)= (void*) realloc(ptr,(nelem)*sizeof(*(ptr)))
\end{verbatim}

\paragraph{Calling C from fortran}
PLUMED assumes that fortran compilers add a single \texttt{\_} at the end of C routines. You
may get an error message as :
\begin{verbatim}
ERROR: Undefined symbol: .init_metadyn
ERROR: Undefined symbol: .meta_force_calculation
\end{verbatim}
eliminate the \texttt{\_} from the definition of init\_metadyn and meta\_force\_calculation, i. e.
change at line 529
\begin{verbatim}
void meta_force_calculation_(real *cell, int *istep, real *xxx, real *yyy, real *zzz, 
\end{verbatim}
with
\begin{verbatim}
void meta_force_calculation(real *cell, int *istep, real *xxx, real *yyy, real *zzz,    
\end{verbatim},
and at line 961
\begin{verbatim}
  void init_metadyn_(int *atoms, real *ddt, real *mass, 
  void init_metadyn_(int *atoms, real *ddt, real *mass, 
\end{verbatim}

\end{document}
