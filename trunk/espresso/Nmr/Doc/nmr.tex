\documentclass[11pt, a4paper]{article}
\usepackage{fullpage}
\usepackage{verbatim}
\usepackage{amsmath}

\title{Implementation of the calculation of NMR parameters in PWSCF}
\author{Mickael Profeta}

\begin{document}

\maketitle 

\section{Introduction}
Nuclear Magnetic Resonance (NMR) spectroscopy is widely used in
structural chemistry and in solid states studies. Experimental spectra
can often be difficult to analyse due to the overlap of the
contribution of the different atoms and the poor signal/noise
ratio. Thanks to the work of C.J. Pickard and F. Mauri
\cite{PRB_GIPAW}, it is possible to calculate the NMR parameters
thanks to a DFT calculation with pseudopotentials.

One of the main issue of this calculation is the determination of the
'all-electron' response through a pseudopential calculation. NMR
parameters are seen on the position of the nuclei.  In a
pseudopotential calculation, the potential in this region, close to
the nuclei, are modified to decrease the number of plane waves used to
converge the calculation. In order to obtain the 'all-electron'
response, it is necessary to reconstruct the results from the
pseudopotential calculation. For the calculation of the electric field
gradient (efg) tensor, we use the PAW method of Bl\"ochl \cite{blochl}
to reconstruct the 'all-electron' results, whereas in the case of the
chemical shift tensor, it is necessary to use an extension of the
previous method due to the presence of the magnetic field, the GIPAW
method\cite{PRB_GIPAW}.

The goal of this work is to implement the calculation of the efg and
the chemical shift tensor with the appropriate reconstruction in
PWSCF. This can be separated in three parts:
\begin{itemize}
\item The atomic part: In order to calculate the reconstruction
element, it is necessary to have the 'all-electron' and
pseudo-wavefunction of each atom in the structure. I then modify the
{\it atomic} code in PWSCF to produce a reconstruction file with the
necessary informations.
\item The calculation of the efg tensor: In order to have the
quadrupolar parameters of the NMR spectra of nuclei of spin over 1/2,
I implement the calculation of the electric field gradient tensor with
the reconstruction of the all-electron response using the PAW
method. This is done as a post-processing program, after a scf
calculation, in the \verb+efg.x+ program.
\item The calculation of the chemical shift tensor: I implement the
calculation of the chemical shift tensor with the GIPAW
reconstruction. This calculation is divided in three parts: After a
SCF calculation, it is necessary to do a non-SCF calculation to
calculate the wavefunction on the k-point k+q, where q is a little
displacement in the three cartesian directions. The third part is the
calculation of the induced current which leads to the chemical shift
tensor, including the GIPAW reconstruction. This is done in the
program \verb+nmr.x+ in the directory Nmr of the PWSCF distribution.
\end{itemize}

\section{Atomic part}

In order to reconstruct the 'all-electron' response from a
pseudopotential calculation, we use the projected augmented wave
method (PAW) of Bl\"ochl. This projectors are constructed with the
pseudo and all-electron atomic wavefunctions corresponding to each
pseudopotentials. It is then necessary to generate these wavefunctions
during the generation of the pseudopotentials.  At the present state,
only norm conserving pseudopotentials are considered.

For an atom, and for each atomic momentum, it is necessary to have two
projectors. Thus we must produce two 'all-electron' and pseudo
wavefunctions for each orbital momentum. To do that, we use the
valence and the first excited wavefunction of each orbital momentum
(for example $2s$ and $3s$ for the $s$ orbital momentum of oxygen). If
necessary, it is possible to use ionised configuration of the atom in
order to have the excited states still bounded.

To produce these wavefunctions, we use the 'test' part of the atomic
code which can generate the needed wavefunctions. An extra flag
"\verb+file_recon+" was added in the input file. When present, the name of
the file after "\verb+file_recon+" is used to write the all-electron and
pseudo wavefunctions in a format close to the UPF format. The
subroutine in charge of the generation of the "\verb+file_recon+" file is
\verb+write_paw_recon.f90+ in the atomic directory. Notice that we check
that the maximum of the all-electron and pseudo wavefunction are
positives.\\

Structure of the "Recon"file:
\begin{verbatim}
<PP_PAW> 
  4 ! Number of the wavefunctions (couple all-electron/pseudo) in the file 
</PP_PAW>

<PP_REC> 
<PP_kbeta> 
  929       ! Number of point in the radial grid
</PP_kbeta> 
<PP_L> 
  0         ! Value of l 
</PP_L> 
<PP_REC_AE> 
0.0 0.0 0.0 0.0  ! The all-electron wavefunction ...  
</PP_REC_AE> 
<PP_REC_PS> 
0.0 0.0
0.0 0.0     ! The pseudo wavefunction ...  
</PP_REC_PS> 
</PP_REC>

<PP_REC> ...
\end{verbatim}

To produce the pseudopotential and the reconstruction, it is advised
to proceed in two steps:
\begin{itemize}
\item first produce the pseudopotential;\\

\verbatiminput{o_nc.in}

\item then produce the \verb+recon_file+ with the same configuration and
cutoff radii, but with the excited states in the all-electron and test
part. \\

\verbatiminput{o_nc_paw.in}

\end{itemize}

\section{efg tensor}

Nuclei with spin greater than 1/2 have a quadrupolar contribution in
NMR spectra. This interaction is responsible for the shape and the
enlargement of the NMR signal of these nuclei. This quadrupolar
interaction is directly related with the electric field gradient (efg)
tensor on the nuclei. Thus in order to calculate these quadrupolar
parameters, we should calculate the efg tensor.

The electric field gradient tensor $\overset{\leftrightarrow}{G}$ is
calculated as the second derivative of the charge density in respect
to the position:

\begin{equation}
 G_{\alpha\beta}({\bf r})=\int\ d^3r^{\prime} \frac{n({\bf
    r^{\prime}})}{|{\bf r}-{\bf
    r}^{\prime}|^3}\left[\delta_{\alpha\beta}
    -3\frac{(r_{\alpha}-r_{\alpha}^{\prime})
    (r_{\beta}-r_{\beta}^{\prime})}{|{\bf r}-{\bf
    r}^{\prime}|^2}\right].
\label{eq_efg} 
\end{equation}

This tensor can be divided in three contribution:
\begin{itemize}
\item The ionic contribution: Contribution of the charge of the frozen
ion, considered as point charge and calculated thanks to an ewald sum.
\item The pseudopotenital contribution: The scf calculation compute
the charge density corresponding to the pseudopotential
calculation. This charge density leads to an efg tensor thanks to
equation (\ref{eq_efg}) where $n({\bf r})$ is replaced by the charge
computed by the scf calculation.
\item The reconstruction contribution: Thanks to PAW method, we can
calculate the contribution of the core region and reconstruct tha
all-electron calculation.
\end{itemize}

\subsection{The ionic contribution}

The contribution of the frozen ions (charge of the nuclei + core
electrons = valence charge) considered as point charge, is calculated
thanks to an ewald sum in the subroutine \verb+/PW/ewald_dipole.f90+.

\begin{itemize}
\item In the real space:
\begin{eqnarray}
G(\alpha, \beta, at) & = & e^2 \sum_{at'} Z_{at'} \left[
\frac{3}{\Delta r^3}\ {\rm erfc}(\eta \Delta r) + \right. \nonumber \\
&& \left. \left(\frac{6 \eta}{\sqrt{\pi}} \frac{1}{\Delta r^2} +
\frac{4 \eta^3}{\sqrt{\pi}} \right) {\rm e}^{-(\eta \Delta r)^2}
\right] \left(\frac{(r_{at'_\alpha} - r_{at_\alpha}) (r_{at'_\beta} -
r_{at_\beta})} {\Delta r^2} - \frac{1}{3} \delta_{\alpha \beta}
\right)
\end{eqnarray}

where $at$ and $at'$ are for the atomic sites, $\Delta r = |{\bf
r}_{at'} - {\bf r}_{at}|$.

\item In the reciprocal space:
\begin{eqnarray}
G(\alpha, \beta, at) & = & e^2 \frac{4 \pi}{V} \sum_{G \ne 0} -
\frac{G_\alpha G_\beta}{G^2} {\rm e}^{\frac{-G^2}{4\eta^2}} \left(
\sum_{at'} Z_{at'}\ {\rm e}^{-i{\bf G}\cdot{\bf r_{at'}}} \right) {\rm
e}^{-i{\bf G}\cdot{\bf r_{at}}}
\end{eqnarray}

\end{itemize}
 
\subsection{The pseudopotentiel contribution}

The electronic charge density is calculated in a SCF calculation. In
order to get the contribution of this charge density to the efg tensor
\begin{itemize}
\item the charge density is transformed in the reciprocal space thanks
to a FFT
\item it is multiply by $(G_\alpha G_\beta)/{G^2}$
\item fourier transformed on the atomic position
\end{itemize}

This is the main part of the subroutine \verb+do_efg+ in the file
\verb+/PP/efg.f90+.

\subsection{The reconstruction part}

The PAW method is based on projectors constructed from the atomic
pseudo and all-electron wavefunctions. Their use is very similar to
the treatment of ultrasoft(US) pseudopotential, and I copy the
structure used to deal with US pseudopotentials for the PAW
projectors.
\begin{itemize}
\item The module \verb+paw+ in the file \verb+/PW/paw.f90+ regroups
the necessary variables for the paw reconstruction. Each variable is
prefixed with paw\_ .
\item This module contains two subroutines \verb+paw_wfc_init+ which
initializes the default values for wavefunction structure and
\verb+read_recon+ which reads the reconstruction file, given in the
input file, for each atom.
\item The subroutine \verb+/PW/init_paw_1.f90+ initializes the
projectors and construct the part which does not depend on k-point. In
particular, when there is more than one wavefuntion for a value of
$\ell$, the wavefunctions are orthogonalised.  Then, each wavefunction
are multiply by a "step function" which goes smoothly to zero from
$r_s$ to $r_c$, where $r_c$ is the core region, usually chosen
slightly bigger than the position of the maximum of the wavefunction,
and $r_s$ is here chosen to be equal to $2/3 r_c$.
\item The subroutine \verb+/PW/init_paw_2.f90+ is very similar to
\verb+/PW/init_us_2.f90+, the only difference is that one acts on US
projectors, the other on PAW.
\item The subroutine \verb+efg_correction+ in the file
\verb+PP/efg.f90+ is the one in charge of the calculation of the
reconstruction term.
\end{itemize}

The reconstruction term to be calculated is:

\begin{eqnarray}
G_{\alpha\beta}^{{\rm El},{\rm corr}} ({\bf R})
&=&2\sum_{o,J,J^{\prime}} \langle\tilde{\psi}_o^{(0)}|\tilde{p}_{{\bf
R},J} \rangle [\langle \phi_{{\bf R},J} | \frac{1} { r^3
}[\delta_{\alpha\beta} -3 \frac{r_\alpha r_\beta} { r^2 }]| \phi_{{\bf
R},J^{\prime}} \rangle \nonumber \\ &&- \langle \tilde{\phi}_{{\bf
R},J} | \frac{1} { {r} ^3 } [\delta_{\alpha\beta} -3 \frac{r_\alpha
r_\beta} { r^2 }]| \tilde{\phi}_{{\bf R},J^{\prime}} \rangle ] \langle
\tilde{p}_{{\bf R},J^{\prime}}|\tilde{\psi}_o^{(0)}\rangle
\label{G_recons}.
\end{eqnarray}

where $|\tilde{\psi}_o^{(0)}\rangle$ is the wavefunction found in the
SCF calculation, $|\tilde{p}_{{\bf R},J} \rangle$ is the PAW projector
for the atom in position R and quantum number J (J stands for the
orbital momentum, and an extra number in the case there is more than
one atomic wavefunction by value of $\ell$). $|\phi_{{\bf
R},J^{\prime}} \rangle$ is the atomic all-electron wavefunction, and
$|\tilde{\phi}_{{\bf R},J^{\prime}} \rangle$ the pseudo atomic
wavefunction.

It can be separated into two parts, a term which depend only on the
radial term and can be computed for each atom
\begin{equation}
\langle R_{{\bf R},J}|\frac{1}{r^3}|R_{{\bf R},J'} \rangle -\langle
\tilde R_{{\bf R},J}|\frac{1}{r^3}|\tilde R_{{\bf R},J'} \rangle
\end{equation}

and a term with the spherical harmonics, the central operator can be
written as $Y_{2m}$.

We obtain the terms for the different values of $m= -2, -1, 0, 1,
2$. Linear combination can be done to obtain the tensor with matrix
element $G_{\alpha\beta}^{{\rm El},{\rm corr}}$ ($\alpha$, $\beta$ are
the cartesian coordinates.\\


The efg tensor is the sum of these three contributions. It is then
symmetrize according to the symmetry of the system.

The tensor obtained, is the all-electron electric field gradient
tensor. If we call $V_{xx}, V_{yy}$ and $V_{zz}$ the three eigenvalues
of this tensor, $|V_{xx}|<|V_{yy}|<|V_{zz}|$ we can calculate the
quadrupolar parameters of the NMR spectroscopy:

 $$ Cq = \frac{e\,Q\,V_{zz}}{h} $$
 
 $$ \eta=\frac{V_{yy}-V_{xx}}{V_{zz}} $$

where $e$ is the electron charge, $h$ the Planck constant and $Q$ the
quadrupolar momentum of the nuclei.\\

Example of input file for the efg.x program:
\verbatiminput{quartz.efg.in}

An example of calculation of the efg tensor is in example24 of PWSCF
distribution.

\section{chemical shift tensor}

A uniform external magnetic field {\bf B} applied to a sample induces
an electric current. This current is proportionnal to the external
field {\bf B} and is the first ordered-induced current ${\bf
  j}^{(1)}({\bf r})$. This current produces a non uniform magnetic
field ${\bf B}_{in}^{(1)}$:
\begin{equation}
{\bf B}_{in}^{(1)}({\bf r})=\frac{1}{c}\int d^3r'\ {\bf  j}^{(1)}({\bf
  r'}) \frac{{\bf r}-{\bf r}'}{|{\bf r}-{\bf r}'|^3}
\label{Bin}
\end{equation}

The chemical shift tensor is defined as the ratio between the induces
magnetic field and the external uniform magnetic field:

\begin{equation}
{\bf B}_{in}^{(1)}({\bf r})=-\overset{\leftrightarrow}{\sigma}({\bf
  r}){\bf B}
\end{equation}

The isotropic chemical shift is one third af the trace of the chemical
shift tensor.

The goal of the calculation is to calculate the all-electron current,
in order to deduce the chemical shift tensor from eq (\ref{Bin}).

\subsection{bare current}

In opposition with the diamagnetic and paramagnetic reconstruction
term, the bare current is the term directly formed by the
pseudopotential wavefunction.

As explained in reference \cite{PRB_GIPAW}, in the case of infinitely
periodic system, the bare induced field can be computed in the
reciprocal space:

\begin{equation}
{\bf B}_{bare}^{(1)}({\bf G})= \frac{4\pi}{c}\frac{i{\bf G}\times{\bf
    j}_{bare}^{(1)}(\bf G)}{G^2}
\end{equation}

equation (61) of reference \cite{PRB_GIPAW}.

For ${\bf G}=0$ previous equation can not be applied, the ${\bf G}=0$
component is:

\begin{equation}
{\bf B}_{in}^{(1)}({\bf G}=0)=
\frac{8\pi}{3}\overset{\leftrightarrow}{\chi}{\bf B}
\end{equation}

equation (62) of reference \cite{PRB_GIPAW}.
Where $\chi$ is macroscopic magnetic susceptibility. We do not take
into account the contribution of the correction current for the
evaluation of $\chi$ which is the contribution from ${\bf
  j}_{bare}^{(1)}$.

An approximation of the evaluation of this susceptibility is:

\begin{equation}
\overset{\leftrightarrow}{\chi}=\frac{\overset{\leftrightarrow}{F}(q)
  -2 \overset{\leftrightarrow}{F}(0)
  +\overset{\leftrightarrow}{F}(-q)} {q^2}
\label{chi}
\end{equation}
equation (64) of reference \cite{PRB_GIPAW}.

where $q$ is a little displacement of the k-point in the 3 cartesian
directions, and $F_{ij}(q)=(2-\delta_{ij}) Q_{ij}(q)$

\begin{equation}
\overset{\leftrightarrow}{Q}(q)=-\sum_{i=x,y,z} \sum_{o,k} {\rm
  Re}\left[\frac{1}{c^2N_kV_c} 
\langle u^{(0)}_{0,k}|{\bf e}_i\times(-i\nabla +
  k)\mathcal{G}_{k+q}(\epsilon_{o,k}){\bf e}_i\times v_{k+q,k}|
u^{(0)}_{0,k} \rangle \right]
\label{Qchi}
\end{equation}

equation (65) of reference \cite{PRB_GIPAW}.

$V_c$ is the unit cell volume $N_k$ the number of k-points and ${\bf
  e}_i$ are unit vectors in the three cartesian directions.

$|u^{(0)}_{0,k} \rangle$ is the ground state electronic Bloch
wavefunction of occupied state $o$ and k-point $k$.

$ v_{k+q,k}= -i \nabla + k + \frac{1}{i}[{\bf r}, V_{k+q,k}^{nl}] $
where $V_{k+q,k}^{nl}$ is the generalized non local potential

\begin{equation}
V_{k+q,k}^{nl}= \sum_{I,J} D_{I,J} |\beta_I^{k+q} \rangle \langle
\beta_J^k |
\end{equation}


The expression of the bare current is

\begin{equation}
{\bf j}_{bare}^{(1)}({\bf r})=\frac{S_{bare}({\bf r},q)-S_{bare}({\bf
    r},-q)} {2q}
\label{jbare}
\end{equation}

Equation (45) of reference \cite{PRB_GIPAW}, where

\begin{equation}
S_{bare}({\bf r},q)=\frac{2}{cN_k}\sum_{i=x,y,z}\sum_{0,k} {\rm
  Re}\left[ \frac{1}{i} \langle u_{o,k}^(0)|J^p_{k,k+q}({\bf r})
\mathcal{G}_{k+q}(\epsilon_{o,k}){\bf B}\times{\bf e}_i\cdot v_{k+q,k} 
|u_{o,k}^(0)\rangle \right]
\end{equation}

Equation (52) of reference \cite{PRB_GIPAW}, and

\begin{equation}
J^p_{k,k+q}({\bf r})=\frac{(-i\nabla+k)|{\bf f}\rangle\langle{\bf r}|
+ |{\bf f}\rangle\langle{\bf r}|(-i\nabla+k+q)}{2}
\end{equation}

Equation (51) of reference \cite{PRB_GIPAW}.

\subsection{Description of a nmr calculation}

To evaluate the bare contribution to the chemical shift, equation
(\ref{chi}) and (\ref{jbare}) must be calculated. In those expression
appears the Bloch electronic function in the occupied states on
k-points ${\bf k}$ and ${\bf k+q}$. Indeed the calculation of the nmr
tensor is divided in three parts:
\begin{itemize}
\item an scf calculation to have the occupied states on k-point ${\bf
  k}$;
\item a non-scf calculation to evaluate the occupied states on k-point
  ${\bf k+q}$. q is a little displacement of the k-point in the three
  cartesian directions. To perform this calculation, we used a flag
  created for the raman calculation which generate automatically the
  different k-point giving the norm of the displacement
  \verb+b_length+.\\

  Example of input file for the non-scf calculation:

  \verbatiminput{inp.nscf}

  the wavefunctions are stored in the .wfc file in this order:

\begin{verbatim}
    k(   1) = (  -0.5000000  -0.5000000  -0.5000000), wk =   2.0000000
    k(   2) = (  -0.4990000  -0.5000000  -0.5000000), wk =   0.0000000
    k(   3) = (  -0.5000000  -0.4990000  -0.5000000), wk =   0.0000000
    k(   4) = (  -0.5000000  -0.5000000  -0.4990000), wk =   0.0000000
    k(   5) = (  -0.5010000  -0.5000000  -0.5000000), wk =   0.0000000
    k(   6) = (  -0.5000000  -0.5010000  -0.5000000), wk =   0.0000000
    k(   7) = (  -0.5000000  -0.5000000  -0.5010000), wk =   0.0000000
\end{verbatim}


\item the nmr calculation in itself. All the subroutines are in a
  directory \verb+Nmr+ of the pwscf CVS. The main program is
  \verb+nmr.f90+. In order to compile it, you should do a
  \verb+make all+ in  this directory. \\

  Example of input for the nmr calculation:

  \verbatiminput{inp.nmr}
  
\end{itemize}

\subsection{description of the code}

\begin{itemize}
\item $|\tilde{u}_{o,k}^{(0)}\rangle=v_{k+q,k}|u_{o,k}^{(0)}\rangle$
  is calculated in subroutine \verb+take_nloc_k_kq+

$\frac{1}{i}[{\bf r},V^{nl}_{k+q,k}]$ is calculated as the derivative
    of the non-local potential with respect to $k$.

\begin{equation}
\frac{1}{i}[{\bf r},V^{nl}_{k+q,k}|u_{o,k}^{(0)}\rangle = \sum_{I,J}
  D_{I,J} \left [ |\beta_I^{k+q}\rangle \frac{\rm d}{\rm dk} \langle
    \beta_J^{k}|u_{o,k}^{(0)}\rangle +  \frac{\rm d}{\rm
      dk}|\beta_I^{k+q} \rangle
  \langle\beta_J^{k}|u_{o,k}^{(0)}\rangle \right]
\end{equation}

In the original code (\verb+paratec+) it was calculated by a numerical
difference, whereas here we try to use an analytical expression as in
\verb+PH/dvpsi_e.f90+. 

\item The subroutine \verb+grad.f90+ simply multiply the wavefunction by
  ${\bf k + G}$.

\item The linear response
  $\mathcal{G}_{k+q}(\epsilon_{o,k})|\tilde{u}_{o,k}^{(0)}\rangle$ is
  calculated in subroutine \verb+solve_cg.f90+ using the conjugate
  gradient in \verb+PH/cgsolve_all.f90+.

\item The subroutine \verb+add_j_bare.f90+ apply the operator $|{\bf
  r}\rangle \langle {\bf r}|$ used in (\ref{jbare}) thanks to a FFT.

\item The subroutine \verb+magnetic_kkterm.f90+ calculate $Q(0)$ in
  (\ref{Qchi}). 

\end{itemize}

Each k-points are calculated separately in the most external loop.

At present time, the calculation with k-point k (\verb+magnetic_kkterm+)
seem to give resonnable output, but the calculation of $\chi$ with
$F(q)$ and $F(-q)$ does not works properly when generalized with
$k+q$. It seems that \verb+take_nloc_k_kq+ has some bugs, but I can
not discover it yet....    

\begin{thebibliography}{1}

\bibitem{PRB_GIPAW}
Chris~J. Pickard and Francesco Mauri.
\newblock All-electron magnetic response with pseudopotentials: Nmr chemical
  shifts.
\newblock {\em Phys. Rev. B}, 63:245101, 2001.

\bibitem{blochl}
M.~Profeta, F.~Mauri, and C.J. Pickard.
\newblock Accurate first principles prediction of $^{17}$o nmr parameters in
  sio$_2$ : assignment of the zeolite ferrierite spectrum.
\newblock {\em J. Am. Chem. Soc.}, 125:541, 2003.

\end{thebibliography}

\end{document}
    
